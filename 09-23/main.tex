\documentclass[a4paper,11pt]{article}
\usepackage[a4paper, margin=8em]{geometry}

% usa i pacchetti per la scrittura in italiano
\usepackage[french,italian]{babel}
\usepackage[T1]{fontenc}
\usepackage[utf8]{inputenc}
\frenchspacing 

% usa i pacchetti per la formattazione matematica
\usepackage{amsmath, amssymb, amsthm, amsfonts}

% usa altri pacchetti
\usepackage{gensymb}
\usepackage{hyperref}
\usepackage{standalone}

\usepackage{colortbl}

\usepackage{xstring}
\usepackage{karnaugh-map}

% imposta il titolo
\title{Appunti Reti Informatiche}
\author{Luca Seggiani}
\date{2025}

% imposta lo stile
% usa helvetica
\usepackage[scaled]{helvet}
% usa palatino
\usepackage{palatino}
% usa un font monospazio guardabile
\usepackage{lmodern}

\renewcommand{\rmdefault}{ppl}
\renewcommand{\sfdefault}{phv}
\renewcommand{\ttdefault}{lmtt}

% circuiti
\usepackage{circuitikz}
\usetikzlibrary{babel}

% testo cerchiato
\newcommand*\circled[1]{\tikz[baseline=(char.base)]{
            \node[shape=circle,draw,inner sep=2pt] (char) {#1};}}

% disponi il titolo
\makeatletter
\renewcommand{\maketitle} {
	\begin{center} 
		\begin{minipage}[t]{.8\textwidth}
			\textsf{\huge\bfseries \@title} 
		\end{minipage}%
		\begin{minipage}[t]{.2\textwidth}
			\raggedleft \vspace{-1.65em}
			\textsf{\small \@author} \vfill
			\textsf{\small \@date}
		\end{minipage}
		\par
	\end{center}

	\thispagestyle{empty}
	\pagestyle{fancy}
}
\makeatother

% disponi teoremi
\usepackage{tcolorbox}
\newtcolorbox[auto counter, number within=section]{theorem}[2][]{%
	colback=blue!10, 
	colframe=blue!40!black, 
	sharp corners=northwest,
	fonttitle=\sffamily\bfseries, 
	title=Teorema~\thetcbcounter: #2, 
	#1
}

% disponi definizioni
\newtcolorbox[auto counter, number within=section]{definition}[2][]{%
	colback=red!10,
	colframe=red!40!black,
	sharp corners=northwest,
	fonttitle=\sffamily\bfseries,
	title=Definizione~\thetcbcounter: #2,
	#1
}

% disponi codice
\usepackage{listings}
\usepackage[table]{xcolor}

\definecolor{codegreen}{rgb}{0,0.6,0}
\definecolor{codegray}{rgb}{0.5,0.5,0.5}
\definecolor{codepurple}{rgb}{0.58,0,0.82}
\definecolor{backcolour}{rgb}{0.95,0.95,0.92}

\lstdefinestyle{codestyle}{
		backgroundcolor=\color{black!5}, 
		commentstyle=\color{codegreen},
		keywordstyle=\bfseries\color{magenta},
		numberstyle=\sffamily\tiny\color{black!60},
		stringstyle=\color{green!50!black},
		basicstyle=\ttfamily\footnotesize,
		breakatwhitespace=false,         
		breaklines=true,                 
		captionpos=b,                    
		keepspaces=true,                 
		numbers=left,                    
		numbersep=5pt,                  
		showspaces=false,                
		showstringspaces=false,
		showtabs=false,                  
		tabsize=2
}

\lstdefinestyle{shellstyle}{
		backgroundcolor=\color{black!5}, 
		basicstyle=\ttfamily\footnotesize\color{black}, 
		commentstyle=\color{black}, 
		keywordstyle=\color{black},
		numberstyle=\color{black!5},
		stringstyle=\color{black}, 
		showspaces=false,
		showstringspaces=false, 
		showtabs=false, 
		tabsize=2, 
		numbers=none, 
		breaklines=true
}


\lstdefinelanguage{assembler}{ 
  keywords={AAA, AAD, AAM, AAS, ADC, ADCB, ADCW, ADCL, ADD, ADDB, ADDW, ADDL, AND, ANDB, ANDW, ANDL,
        ARPL, BOUND, BSF, BSFL, BSFW, BSR, BSRL, BSRW, BSWAP, BT, BTC, BTCB, BTCW, BTCL, BTR, 
        BTRB, BTRW, BTRL, BTS, BTSB, BTSW, BTSL, CALL, CBW, CDQ, CLC, CLD, CLI, CLTS, CMC, CMP,
        CMPB, CMPW, CMPL, CMPS, CMPSB, CMPSD, CMPSW, CMPXCHG, CMPXCHGB, CMPXCHGW, CMPXCHGL,
        CMPXCHG8B, CPUID, CWDE, DAA, DAS, DEC, DECB, DECW, DECL, DIV, DIVB, DIVW, DIVL, ENTER,
        HLT, IDIV, IDIVB, IDIVW, IDIVL, IMUL, IMULB, IMULW, IMULL, IN, INB, INW, INL, INC, INCB,
        INCW, INCL, INS, INSB, INSD, INSW, INT, INT3, INTO, INVD, INVLPG, IRET, IRETD, JA, JAE,
        JB, JBE, JC, JCXZ, JE, JECXZ, JG, JGE, JL, JLE, JMP, JNA, JNAE, JNB, JNBE, JNC, JNE, JNG,
        JNGE, JNL, JNLE, JNO, JNP, JNS, JNZ, JO, JP, JPE, JPO, JS, JZ, LAHF, LAR, LCALL, LDS,
        LEA, LEAVE, LES, LFS, LGDT, LGS, LIDT, LMSW, LOCK, LODSB, LODSD, LODSW, LOOP, LOOPE,
        LOOPNE, LSL, LSS, LTR, MOV, MOVB, MOVW, MOVL, MOVSB, MOVSD, MOVSW, MOVSX, MOVSXB,
        MOVSXW, MOVSXL, MOVZX, MOVZXB, MOVZXW, MOVZXL, MUL, MULB, MULW, MULL, NEG, NEGB, NEGW,
        NEGL, NOP, NOT, NOTB, NOTW, NOTL, OR, ORB, ORW, ORL, OUT, OUTB, OUTW, OUTL, OUTSB, OUTSD,
        OUTSW, POP, POPL, POPW, POPB, POPA, POPAD, POPF, POPFD, PUSH, PUSHL, PUSHW, PUSHB, PUSHA, 
				PUSHAD, PUSHF, PUSHFD, RCL, RCLB, RCLW, MOVSL, MOVSB, MOVSW, STOSL, STOSB, STOSW, LODSB, LODSW,
				LODSL, INSB, INSW, INSL, OUTSB, OUTSL, OUTSW
        RCLL, RCR, RCRB, RCRW, RCRL, RDMSR, RDPMC, RDTSC, REP, REPE, REPNE, RET, ROL, ROLB, ROLW,
        ROLL, ROR, RORB, RORW, RORL, SAHF, SAL, SALB, SALW, SALL, SAR, SARB, SARW, SARL, SBB,
        SBBB, SBBW, SBBL, SCASB, SCASD, SCASW, SETA, SETAE, SETB, SETBE, SETC, SETE, SETG, SETGE,
        SETL, SETLE, SETNA, SETNAE, SETNB, SETNBE, SETNC, SETNE, SETNG, SETNGE, SETNL, SETNLE,
        SETNO, SETNP, SETNS, SETNZ, SETO, SETP, SETPE, SETPO, SETS, SETZ, SGDT, SHL, SHLB, SHLW,
        SHLL, SHLD, SHR, SHRB, SHRW, SHRL, SHRD, SIDT, SLDT, SMSW, STC, STD, STI, STOSB, STOSD,
        STOSW, STR, SUB, SUBB, SUBW, SUBL, TEST, TESTB, TESTW, TESTL, VERR, VERW, WAIT, WBINVD,
        XADD, XADDB, XADDW, XADDL, XCHG, XCHGB, XCHGW, XCHGL, XLAT, XLATB, XOR, XORB, XORW, XORL},
  keywordstyle=\color{blue}\bfseries,
  ndkeywordstyle=\color{darkgray}\bfseries,
  identifierstyle=\color{black},
  sensitive=false,
  comment=[l]{\#},
  morecomment=[s]{/*}{*/},
  commentstyle=\color{purple}\ttfamily,
  stringstyle=\color{red}\ttfamily,
  morestring=[b]',
  morestring=[b]"
}

\lstset{language=assembler, style=codestyle}

% disponi sezioni
\usepackage{titlesec}

\titleformat{\section}
	{\sffamily\Large\bfseries} 
	{\thesection}{1em}{} 
\titleformat{\subsection}
	{\sffamily\large\bfseries}   
	{\thesubsection}{1em}{} 
\titleformat{\subsubsection}
	{\sffamily\normalsize\bfseries} 
	{\thesubsubsection}{1em}{}

% tikz
\usepackage{tikz}

% float
\usepackage{float}

% grafici
\usepackage{pgfplots}
\pgfplotsset{width=10cm,compat=1.9}

% disponi alberi
\usepackage{forest}

\forestset{
	rectstyle/.style={
		for tree={rectangle,draw,font=\large\sffamily}
	},
	roundstyle/.style={
		for tree={circle,draw,font=\large}
	}
}

% disponi algoritmi
\usepackage{algorithm}
\usepackage{algorithmic}
\makeatletter
\renewcommand{\ALG@name}{Algoritmo}
\makeatother

% disponi numeri di pagina
\usepackage{fancyhdr}
\fancyhf{} 
\fancyfoot[L]{\sffamily{\thepage}}

\makeatletter
\fancyhead[L]{\raisebox{1ex}[0pt][0pt]{\sffamily{\@title \ \@date}}} 
\fancyhead[R]{\raisebox{1ex}[0pt][0pt]{\sffamily{\@author}}}
\makeatother

\begin{document}
% sezione (data)
\section{Lezione del 23-09-25}

% stili pagina
\thispagestyle{empty}
\pagestyle{fancy}

% testo
\subsection{Introduzione}
Il corso si pone di presentare le nozioni di base sulle reti informatiche, le tecnologie di rete più diffuse, i protocolli Internet e lo sviluppo di applicazioni distribuite \textit{client-server} e \textit{peer-to-peer} (\textbf{P2P}).

In particolare il programma del corso comprende:
\begin{itemize}
	\item Sviluppo di \textbf{applicazioni} in rete:
		\begin{itemize}
			\item Client-server;
			\item Peer-to-peer.
		\end{itemize}
	\item Reti a \textbf{connessione diretta}:
		\begin{itemize}
			\item Collegamenti punto-punto;
			\item Reti locali.
		\end{itemize}
	\item Reti a \textbf{commutazione di pacchetto};
	\item \textbf{Interconnessione} di reti di tipo diverso;
	\item \textbf{Trasporto} end-to-end e protocolli;
	\item \textbf{Sicurezza};
	\item Reti \textbf{wireless} e \textbf{mobili}, intese come caso particolare delle normali reti \textbf{cablate} (\textit{wired}).
\end{itemize}

\subsubsection{Applicazioni in rete}
Nel dettaglio delle \textit{applicazioni in rete}, vedremo come già detto i paradigmi \textit{client-server} e \textit{peer-to-peer}, di cui possiamo già fare alcuni esempi:
\begin{itemize}
	\item Applicazioni client-server:
		\begin{itemize}
			\item Web;
			\item Trasferimento file;
			\item Posta elettronica;
			\item DNS;
			\item Ecc...
		\end{itemize}
	\item Applicazioni peer-to-peer:
		\begin{itemize}
			\item Ricerca di contenuti;
			\item Torrent;
			\item Telefonia online;
			\item Ecc...
		\end{itemize}
\end{itemize}

In questo ci avvarremo del concetto di \textbf{socket} come primitiva per la gestione della rete dal lato S/O.

\subsubsection{Reti dirette, a commutazione e wireless}
Inizieremo con lo studio di \textit{collegamenti punto-punto}, e quindi di trasferimento affidabile di dati fra 2 punti.
Vedremo poi le reti locali, ad accesso multiplo, e i casi particolari come \textit{Ethernet}.

Vedremo quindi le reti a \textit{commutazione di pacchetto} per la copertura di grandi regioni. Anche qui approfondiremo tecnologie come gli \textit{switch}, ancora \textit{Ethernet}, ecc...

Per quanto riguarda l'\textit{interconnessione di reti} vedremo il protocollo Internet \textbf{IPv4}, il \textbf{routing} (cioè l'\textit{instradamento}) e i protocolli di trasporto (\textbf{UDP} e \textbf{TCP}).

Parleremo anche di reti \textit{wireless} e \textit{mobili}, e quindi di tecnologie come \textbf{WiFi}, le \textbf{reti cellulari}, e reti senza infrastruttura come \textbf{Bluetooth}.

\subsubsection{Sicurezza}
Vedremo poi le minacce alla \textit{sicurezza} e alcune soluzioni che abbiamo a disposizione per mitigarle.
In particolare, tratteremo di \textbf{crittografia} e \textbf{integrità} dei messaggi. 

Nello specifico parleremo di tecnologie a livello applicazione (\textbf{PGP}), a livello trasporto (\textbf{TLS} (usata in \textit{HTTPS})), a livello Internet (\textbf{IP-Sec}) e difese di sicurezza come \textbf{firewall} e \textbf{IDS}.

\subsection{Terminologia}
Iniziamo quindi a definire la terminologia di base usata nel corso, usando Internet come esempio.

\subsubsection{Internet}
La prima domanda che ci poniamo è \textit{"Che cos'è Internet?"}.

\par\medskip
\textbf{\textsf{Visione ingegneristica}} \\
\noindent
Iniziamo col vedere la definizione di Internet agli occhi di un ingegnere che si occupa di reti: 
\begin{itemize}
	\item Si tratterà di una rete che connette miliardi di \textit{dispositivi}, detti \textbf{host} (\textit{"ospiti"}), che eseguono \textit{applicazioni in rete} al cosiddetto \textbf{edge} (\textit{"bordo"}) della rete.
	\item Una visione \textbf{interna} della rete ci dirà invece che è un insieme di \textbf{pacchetti} che viaggiano attraverso infrastruttura (\textit{router}, \textit{switch}), per raggiungere il loro destinatario.
	\item A livello \textbf{fisico} potremmo considerare le connessioni fisiche fra dispositivi, date da cavi, segnali radio, ecc...
	\item Infine, potremo organizzare le \textbf{reti} come collezioni di dispositivi, router e connessioni gestite da determinate organizzazioni.
\end{itemize}

Non è esattamente corretto parlare di \textit{"reti di calcolatori"} in quanto oggi ad essere connessi a Internet sono tutta una gamma di dispositivi non necessariamente orientati al puro \textit{calcolo}: è questo il caso del cosiddetto \textit{Internet of Things} (\textbf{IoT}).

\par\medskip

Possiamo quindi intendere Internet come una "rete di reti", cioè più \textbf{ISP} (\textit{Internet Service Providers}) connessi fra di loro, che a loro volta connettono una gamma dispositivi (host, router, switch, ecc...).

Per governare l'operazione di tali rete si necessita di \textbf{protocolli}, che definiscono il modo in cui si inviano e ricevono messaggi in rete.

In particolare, per quanto riguarda Internet notiamo l'\textbf{IETF} (\textit{Internet Engineering Task Force}), organizzazione che gestisce diversi standard del settore (anche detti \textbf{RFC}, da \textit{Request For Comments memoranda}).

\par\medskip
\textbf{\textsf{Visione utente}} \\
\noindent
Per l'utente, internet sarà un insieme di \textbf{infrastrutture} che forniscono \textbf{servizi} finali, fra cui il Web, telecomunicazioni, streaming, ecc...
Dal punto di vista delle \textbf{applicazioni} in esecuzione sui dispositivi, Internet rappresenterà un'interfaccia di programmazione per consentire la comunicazione fra processi su una o più macchine.
In questo parleremo di \textbf{hook} che permettono alle applicazioni di \textbf{connettersi} a Internet, cioè accedere ad un qualche protocollo di trasporto dei dati. 

\subsubsection{Protocolli}
Un \textbf{protocollo} è una precisa \textit{specifica} del formato secondo il quale due dispositivi in rete si scambiano informazioni.
Solitamente i protocolli si sviluppano in più fasi, successive nel tempo, dove si portano avanti diverse operazioni necessarie alla comunicazione.

Esempi di protocollo sono il \textit{protocollo Internet} \textbf{IPv4}, e il \textit{protocollo di trasporto} \textbf{TCP} usato nel \textit{Web} e visto nel corso di progettazione web.

\subsubsection{Infrastruttura di Internet}
Vediamo più nel dettaglio la struttura di Internet:
\begin{itemize}
	\item 
	Abbiamo detto che all'\textit{edge} di internet ci sono i cosiddetti \textit{host}, cioè i \textbf{client} e i \textbf{server}. Notiamo che non vogliamo riferirci alle macchine fisiche client o server, ma ai \textbf{processi} che si comportano come tali per l'implementazione di un'applicazione distribuita.
	\item
	I dispositivi \textit{terminali} che abbiamo appena nominato accedono ad Internet attraverso le cosiddette \textbf{reti di accesso}, cablate o wireless e basate sulle tecnologie utilizzate (router).

	Per collegare i sistemi terminali ai router si usano \textit{reti residenziali}, \textit{reti di accesso istituzionali} (scuola, lavoro, ecc...), nonché \textit{reti wireless e mobili} (Wifi, o reti come 4G solitamente fornite da privati).
	In questo caso può interessarci la frequenza di trasmissione, in bit al secondo, di una rete di accesso, o se quella rete è ad accesso \textit{condiviso} (pensa WiFi) o \textit{dedicato} (pensa Ethernet).

	Uno standard storico per le reti di accesso è quello della trasmissione sulla linea telefonica su \textbf{DSL} (\textit{Digital Subscriber Line}).
	Negli Stati Uniti si è invece diffuso l'uso dela linea televisiva cablata.
	Oggi, sfruttiamo invece tecnologie come \textbf{ADSL} (\textit{Asymmetric Digital Subscriber Line}) e \textbf{FTTC} (\textit{Fiber To The Cabinet}).
	La differenza principale fra queste è che la linea in ADSL è interamente in rame, sia dalla centrale all'armadio di ripartizione che dall'armadio ripartilinea agli utenti finali, mentre nella linea FTTC si porta il segnale all'armadio attraverso cavi in fibra ottica.
	Lo standard di ultima generazione è \textbf{FTTH} (\textit{Fiber To The Home}), che prevede una linea in fibra ottica anche dall'armadio agli utenti finali.

	Possiamo quindi vedere la rete locale (\textbf{LAN} (\textit{Local Area Network})) di una comune abitazione come composta da un router, connesso a un \textbf{modem} DSL (\textit{modem} deriva da modulatore/demodulatore sulla linea telefonica dei messaggi Internet) o direttamente via cavo ad un altro centro di ripartizione, e ad eventuali dispositivi come \textit{access point WiFi} che offrono la connessione via rete mobile ai dispositivi finali (una cosiddetta \textbf{WLAN}, \textit{Wireless Local Area Network}).

	Altre soluzioni per le comunicazioni wireless sono rappresentati da reti \textbf{cellulari} su larga scala, che sono quelle usate dagli operatori telefonici (tecnologie come \textbf{4G}, ecc...).

	\item 
	Dalle reti di accesso si arriva a Internet attraverso reti interconnesse di router, arrivando quindi alle \textit{reti di reti} di cui stavamo parlando.
\end{itemize}


\end{document}
