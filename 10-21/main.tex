\documentclass[a4paper,11pt]{article}
\usepackage[a4paper, margin=8em]{geometry}

% usa i pacchetti per la scrittura in italiano
\usepackage[french,italian]{babel}
\usepackage[T1]{fontenc}
\usepackage[utf8]{inputenc}
\frenchspacing 

% usa i pacchetti per la formattazione matematica
\usepackage{amsmath, amssymb, amsthm, amsfonts}

% usa altri pacchetti
\usepackage{gensymb}
\usepackage{hyperref}
\usepackage{standalone}

\usepackage{colortbl}

\usepackage{xstring}
\usepackage{karnaugh-map}

% imposta il titolo
\title{Appunti Reti Informatiche}
\author{Luca Seggiani}
\date{2025}

% imposta lo stile
% usa helvetica
\usepackage[scaled]{helvet}
% usa palatino
\usepackage{palatino}
% usa un font monospazio guardabile
\usepackage{lmodern}

\renewcommand{\rmdefault}{ppl}
\renewcommand{\sfdefault}{phv}
\renewcommand{\ttdefault}{lmtt}

% circuiti
\usepackage{circuitikz}
\usetikzlibrary{babel}

% testo cerchiato
\newcommand*\circled[1]{\tikz[baseline=(char.base)]{
            \node[shape=circle,draw,inner sep=2pt] (char) {#1};}}

% disponi il titolo
\makeatletter
\renewcommand{\maketitle} {
	\begin{center} 
		\begin{minipage}[t]{.8\textwidth}
			\textsf{\huge\bfseries \@title} 
		\end{minipage}%
		\begin{minipage}[t]{.2\textwidth}
			\raggedleft \vspace{-1.65em}
			\textsf{\small \@author} \vfill
			\textsf{\small \@date}
		\end{minipage}
		\par
	\end{center}

	\thispagestyle{empty}
	\pagestyle{fancy}
}
\makeatother

% disponi teoremi
\usepackage{tcolorbox}
\newtcolorbox[auto counter, number within=section]{theorem}[2][]{%
	colback=blue!10, 
	colframe=blue!40!black, 
	sharp corners=northwest,
	fonttitle=\sffamily\bfseries, 
	title=Teorema~\thetcbcounter: #2, 
	#1
}

% disponi definizioni
\newtcolorbox[auto counter, number within=section]{definition}[2][]{%
	colback=red!10,
	colframe=red!40!black,
	sharp corners=northwest,
	fonttitle=\sffamily\bfseries,
	title=Definizione~\thetcbcounter: #2,
	#1
}

% disponi codice
\usepackage{listings}
\usepackage[table]{xcolor}

\definecolor{codegreen}{rgb}{0,0.6,0}
\definecolor{codegray}{rgb}{0.5,0.5,0.5}
\definecolor{codepurple}{rgb}{0.58,0,0.82}
\definecolor{backcolour}{rgb}{0.95,0.95,0.92}

\lstdefinestyle{codestyle}{
		backgroundcolor=\color{black!5}, 
		commentstyle=\color{codegreen},
		keywordstyle=\bfseries\color{magenta},
		numberstyle=\sffamily\tiny\color{black!60},
		stringstyle=\color{green!50!black},
		basicstyle=\ttfamily\footnotesize,
		breakatwhitespace=false,         
		breaklines=true,                 
		captionpos=b,                    
		keepspaces=true,                 
		numbers=left,                    
		numbersep=5pt,                  
		showspaces=false,                
		showstringspaces=false,
		showtabs=false,                  
		tabsize=2
}

\lstdefinestyle{shellstyle}{
		backgroundcolor=\color{black!5}, 
		basicstyle=\ttfamily\footnotesize\color{black}, 
		commentstyle=\color{black}, 
		keywordstyle=\color{black},
		numberstyle=\color{black!5},
		stringstyle=\color{black}, 
		showspaces=false,
		showstringspaces=false, 
		showtabs=false, 
		tabsize=2, 
		numbers=none, 
		breaklines=true
}


\lstdefinelanguage{assembler}{ 
  keywords={AAA, AAD, AAM, AAS, ADC, ADCB, ADCW, ADCL, ADD, ADDB, ADDW, ADDL, AND, ANDB, ANDW, ANDL,
        ARPL, BOUND, BSF, BSFL, BSFW, BSR, BSRL, BSRW, BSWAP, BT, BTC, BTCB, BTCW, BTCL, BTR, 
        BTRB, BTRW, BTRL, BTS, BTSB, BTSW, BTSL, CALL, CBW, CDQ, CLC, CLD, CLI, CLTS, CMC, CMP,
        CMPB, CMPW, CMPL, CMPS, CMPSB, CMPSD, CMPSW, CMPXCHG, CMPXCHGB, CMPXCHGW, CMPXCHGL,
        CMPXCHG8B, CPUID, CWDE, DAA, DAS, DEC, DECB, DECW, DECL, DIV, DIVB, DIVW, DIVL, ENTER,
        HLT, IDIV, IDIVB, IDIVW, IDIVL, IMUL, IMULB, IMULW, IMULL, IN, INB, INW, INL, INC, INCB,
        INCW, INCL, INS, INSB, INSD, INSW, INT, INT3, INTO, INVD, INVLPG, IRET, IRETD, JA, JAE,
        JB, JBE, JC, JCXZ, JE, JECXZ, JG, JGE, JL, JLE, JMP, JNA, JNAE, JNB, JNBE, JNC, JNE, JNG,
        JNGE, JNL, JNLE, JNO, JNP, JNS, JNZ, JO, JP, JPE, JPO, JS, JZ, LAHF, LAR, LCALL, LDS,
        LEA, LEAVE, LES, LFS, LGDT, LGS, LIDT, LMSW, LOCK, LODSB, LODSD, LODSW, LOOP, LOOPE,
        LOOPNE, LSL, LSS, LTR, MOV, MOVB, MOVW, MOVL, MOVSB, MOVSD, MOVSW, MOVSX, MOVSXB,
        MOVSXW, MOVSXL, MOVZX, MOVZXB, MOVZXW, MOVZXL, MUL, MULB, MULW, MULL, NEG, NEGB, NEGW,
        NEGL, NOP, NOT, NOTB, NOTW, NOTL, OR, ORB, ORW, ORL, OUT, OUTB, OUTW, OUTL, OUTSB, OUTSD,
        OUTSW, POP, POPL, POPW, POPB, POPA, POPAD, POPF, POPFD, PUSH, PUSHL, PUSHW, PUSHB, PUSHA, 
				PUSHAD, PUSHF, PUSHFD, RCL, RCLB, RCLW, MOVSL, MOVSB, MOVSW, STOSL, STOSB, STOSW, LODSB, LODSW,
				LODSL, INSB, INSW, INSL, OUTSB, OUTSL, OUTSW
        RCLL, RCR, RCRB, RCRW, RCRL, RDMSR, RDPMC, RDTSC, REP, REPE, REPNE, RET, ROL, ROLB, ROLW,
        ROLL, ROR, RORB, RORW, RORL, SAHF, SAL, SALB, SALW, SALL, SAR, SARB, SARW, SARL, SBB,
        SBBB, SBBW, SBBL, SCASB, SCASD, SCASW, SETA, SETAE, SETB, SETBE, SETC, SETE, SETG, SETGE,
        SETL, SETLE, SETNA, SETNAE, SETNB, SETNBE, SETNC, SETNE, SETNG, SETNGE, SETNL, SETNLE,
        SETNO, SETNP, SETNS, SETNZ, SETO, SETP, SETPE, SETPO, SETS, SETZ, SGDT, SHL, SHLB, SHLW,
        SHLL, SHLD, SHR, SHRB, SHRW, SHRL, SHRD, SIDT, SLDT, SMSW, STC, STD, STI, STOSB, STOSD,
        STOSW, STR, SUB, SUBB, SUBW, SUBL, TEST, TESTB, TESTW, TESTL, VERR, VERW, WAIT, WBINVD,
        XADD, XADDB, XADDW, XADDL, XCHG, XCHGB, XCHGW, XCHGL, XLAT, XLATB, XOR, XORB, XORW, XORL},
  keywordstyle=\color{blue}\bfseries,
  ndkeywordstyle=\color{darkgray}\bfseries,
  identifierstyle=\color{black},
  sensitive=false,
  comment=[l]{\#},
  morecomment=[s]{/*}{*/},
  commentstyle=\color{purple}\ttfamily,
  stringstyle=\color{red}\ttfamily,
  morestring=[b]',
  morestring=[b]"
}

\lstset{language=assembler, style=codestyle}

% disponi sezioni
\usepackage{titlesec}

\titleformat{\section}
	{\sffamily\Large\bfseries} 
	{\thesection}{1em}{} 
\titleformat{\subsection}
	{\sffamily\large\bfseries}   
	{\thesubsection}{1em}{} 
\titleformat{\subsubsection}
	{\sffamily\normalsize\bfseries} 
	{\thesubsubsection}{1em}{}

% tikz
\usepackage{tikz}

% float
\usepackage{float}

% grafici
\usepackage{pgfplots}
\pgfplotsset{width=10cm,compat=1.9}

% disponi alberi
\usepackage{forest}

\forestset{
	rectstyle/.style={
		for tree={rectangle,draw,font=\large\sffamily}
	},
	roundstyle/.style={
		for tree={circle,draw,font=\large}
	}
}

% disponi algoritmi
\usepackage{algorithm}
\usepackage{algorithmic}
\makeatletter
\renewcommand{\ALG@name}{Algoritmo}
\makeatother

% disponi numeri di pagina
\usepackage{fancyhdr}
\fancyhf{} 
\fancyfoot[L]{\sffamily{\thepage}}

\makeatletter
\fancyhead[L]{\raisebox{1ex}[0pt][0pt]{\sffamily{\@title \ \@date}}} 
\fancyhead[R]{\raisebox{1ex}[0pt][0pt]{\sffamily{\@author}}}
\makeatother

\begin{document}
% sezione (data)
\section{Lezione del 21-10-25}

% stili pagina
\thispagestyle{empty}
\pagestyle{fancy}

% testo
Riprendiamo la discussione dei protocolli MAC.

\subsubsection{Protocollo FDMA}
Avevamo parlato del protocollo \textit{TDMA}. Adesso vediamone la versione in partizionamento di stretto: l'\textbf{FDMA} (\textit{Frequency Division Multiple Access}).

In questo caso si divide il canale in bande di frequenze. Le bande risultano così ugualmente distribuite fra tutti i nodi, e allo stesso modo il bitrate è diviso fra le bande.

IN questo il FDMA ha gli stessi pro e contro del TDMA: i primi 2 requisiti sono soddisfatti direttamente. La centralizzazione (3) è sempre un problema in quanto qualcuno dovrà almeno definire quali sono le bande. Infine, il protocollo è nuovamente semplice (4) (ammesso che uno conosca la trasformata di Fourier discreta). 

\subsection{MAC ad accesso casuale}
Veniamo quindi a discutere i protocolli ad \textbf{accesso casuale}.
In questo caso ipotizziamo che:
\begin{itemize}
	\item Quando un nodo ha un pacchetto, vuole trasmetterlo al bitrate massimo $R$;
	\item Non c'è alcun tipo di coordinazione \textit{a priori} dei nodi.
\end{itemize}

In questo caso, chiaramente si incorre nel problema delle \textit{collisioni}.

I protocolli \textbf{MAC random access} specificano come:
\begin{itemize}
	\item Rilevare collisioni;
	\item Recuperare dalle collisioni (ad esempio ritrasmettendo con ritardi casuali).
\end{itemize}

Esempi di protcolli MAC random access sono \textbf{ALOHA} (originariamente acronimo \textit{Additive Links On-line Hawaii Area}) e \textbf{CSMA} (\textit{Carrier Sense Multiple Access}).

\subsubsection{Protocollo ALOHA slotted}
Il protocollo \textbf{ALOHA} (\textit{Additive Links On-Line Hawaii Area}) nasce da un progetto di Norman Abramson, per coordinare le comunicazioni fra le sedi dell'università delle Hawaii.

Vediamo la versione di ALOHA detta \textbf{ALOHA slotted} (tempo diviso in \textit{slot}).
Assumiamo che:
\begin{itemize}
	\item Tutti i frame hanno la solita dimensione prefissata;
	\item Il tempo è diviso in slot di dimensioni uguali (corrispondenti al tempo necessario a trasmettere 1 frame);
	\item I nodi iniziano a trasmettere solo all'inizio degli slot;
	\item I nodi sono sincronizzati;
	\item Se 2 o più nodi iniziano a trasmettere all'inizio di uno slot, tutti gli slot rilevano una collisione.
\end{itemize}

La modalità di operazione dei nodi in ALOHA è quindi la seguente:
\begin{itemize}
	\item Quando un nodo ottiene un frame da trasmettere, lo trasmette al prossimo slot;
	\item Se non c'è collisione, la trasmissione va a buon fine e il frame viene trasmesso;
	\item Se c'è collisione, il nodo ritrasmette il frame in ogni slot seguente con un certa probabilità $p$ finché la trasmissione non riesce.
\end{itemize}

Le collisioni vengono verificate controllando se si riceve un ACK al termine della trasmissione: notiamo che questo rende identici (dal punto di vista del trasmettitore) gli errori di trasmissione di ACK e le collisioni. I trasmettitori in ALOHA assumono che ci sia comunque bisogno di ritrasmettere.

I pro dell'approccio sono che ogni nodo può trasmettere al bitrate $R$ completo (1), il sistema è altamente decentralizzato (3) (assunta sincronizzazione), ed è semplice (4).

I contro sono che le collisioni riducono il bitrate effettivo, sprecando slot e quindi impedendo che $M$ nodi possano trasmettere a capacità $R/M$ (3). Inoltre, si possono verificare slot fermi. Infine, notiamo che abbiamo bisogno di sincronizzazione di clock o comunque una sincronia che permetta ai nodi di capire quando gli slot iniziano.

\subsubsection{Efficienza di ALOHA slotted}
Facciamo alcune considerazioni quantitative sull'efficienza del protocollo appena visto.
Assumiamo che $N$ nodi vogliano trasmettere frame negli slot con probabilità $p$.

La probabilità che ogni nodo trasmetta con successo in uno slot sarà:
$$
p_\text{nodo} = p(1 - p)^{N - 1}
$$
per cui la probabilità che un \textit{qualsiasi} nodo trasmetta con successo in uno slot sarà:
$$
p_{\text{nodi}} = N p(1 - p)^{N - 1}
$$

Vogliamo quindi trovare $p^*$ che massimizza $p_\text{nodi}$.
Derivando $p_\text{nodi}$ su $p$ si ha:
$$
\frac{d}{dp} \, p_\text{nodi} = N(1 - p)^{N - 1} - p N (N - 1) (1 - p)^{N - 2}
= N (1 - p)^{N - 2} \left( (1 - p) - p(N - 1) \right)
$$
che imponiamo uguale a zero:
$$
\frac{d}{dp} p_\text{nodi} = N (1 - p)^{N - 2} \left( (1 - p) - p(N - 1) \right) = 0
$$
Ora, $(1 - p)^{N - 2}$ è sempre diverso da 0, salvo il caso $p = 1$ (che però sarebbe troppo facile, significherebbe che tutti i nodi ritrasmettono sempre, e le collisioni sono quindi assicurate).
Inoltre, a meno che la nostra rete non sia banale, $N \neq 0$.
Poniamo quindi $p \neq 1 \implies (1 - p)^{N - 2} \neq 0$, $N \neq 0$ e cancelliamo i rispettivi termini:
$$
N \left( (1 - p) - p(N - 1) \right) = 0 \implies
$$

# finisci!!!

\subsubsection{Protocollo ALOHA pure}
Eliminiamo l'ipotesi della sincronizzazione dal protocollo ALOHA in versione slotted: in questo caso, appena arriva un frame, il nodo inizia subito a trasmettere.
Chiamiamo questo protocollo \textbf{ALOHA pure}.

La probabilità delle collisioni in questo caso incrementa: ogni frame inviato al tempo $t_0$ collide con i frame inviati fra $[t_0 - 1, t_0 + 1]$ (dove l'unita corrisponde allo slot temporale dedicato ad un frame).

\subsubsection{Efficienza di ALOHA pure}
# divertiti coi calcoli

\par\medskip

Abbiamo quindi trovato un protocollo che è molto più semplice e decentralizzato (non richiede sincronizzazione), ma ha un utilizzazione del canale di comuniazione che è di molto minore della versione slotted.

\subsubsection{Protocollo CSMA}
Il protocollo \textbf{CSMA} (\textit{Carrier Sense Multiple Access}) prevede di \textit{ascoltare} prima di trasmettere: in questo caso si può rilevare la condizione del canale condiviso prima di provare ad accedervi.
In particolare, dopo aver ascoltato:
\begin{itemize}
	\item Se il canale è rilevato fermo, si trasmette l'intero frame;
	\item Se il canae è rilevato attivo, si differisce la trasmissione ad un secondo momento.
\end{itemize}

Questo è in qualche modo analogo al modo in cui gli umani usano mezzi condivisi (ad esempio l'etere quando parlano a voce): prima si ascolta cosa dicono gli altri, e poi si parla. La regola fondamentale è \textit{"non interrompere gli altri!"}.

CSMA ha una versione detta \textbf{CSMA/CD}, cioè \textit{CSMA with Collision Detection}.
In questo caso le collisioni vengono rilevate entro qualche tempo limitato.
Le trasmissioni in collisione vengono abortite, riducendo lo spreco del canale.
Questo è piuttosto facile per i mezzi cablati, più difficile per i mezzi wireless. 

Il CSMA/CD è necessario in quanto le collisioni possono comunque accadere dopo l'ascolto del canale (\textit{carrier sensing}): i tempi di propagazione significano infatti che due nodi potrebbero non sentire le trasmissioni appena iniziate l'uno dell'altro.

\subsubsection{Rilevamento collisioni CSMA}
Per rilevare le collisioni si potrebbe pensare di mettere a comparatore il segnale che il nodo sta trasmettendo e quello che sta ricevendo: se il delta è maggiore di qualche soglia, dev'essere che c'è un altra sorgente di segnale e quindi siamo in collisione.

In verità l'approccio effettivamente usato è più semplice: invece di parlare di \textit{segnali}, si parla di \textit{potenze} rilevate sul mezzo di trasmissione. L'approccio è comunque funzionale: se il trasmettitore si aspetta di poter erogare una potenza $P$, rilevando potenze $P^* >> P$ sul mezzo potremo concludere con un certo grado di sicurezza di essere in collisione.

\subsubsection{CSMA su Ethernet}
Vediamo quindi l'implementazione di CSMA sul mezzo Ethernet.
Vediamo cosa fa la \textbf{NIC} (\textit{Network Interface Card}) quando vuole trasmettere un frame.
\begin{enumerate}
	\item La NIC riceve il datagramma dal livello network, e ne crea un frame (siamo a livello datalink);
	\item La NIC ascolta il mezzo (\textit{channel sensing}):
		\begin{itemize}
			\item Se il mezzo è rilevato fermo, inizia con la trasmissione del frame;
			\item Se il mezzo è rilevato attivo, si aspetta finché non è fermo, e quindi si trasmette.
		\end{itemize}
	\item Se la NIC riesce a trasmettere l'intero frame senza collisioni, abbiamo finito;
	\item Se si verificano altre trasmissioni mentre si trasmette, cioè una \textit{collisione} (vedi sezione sopra), si abortisce la trasmissione e invia il cosiddetto segnale di \textbf{jam}: questo è un segnale a potenza più alta della media che ha lo scopo di avvisare gli altri trasmettitori che una collisione si è verificata;
	\item Dopo aver abortito, la NIC entra in un \textbf{backoff esponenziale}:
		\begin{itemize}
			\item Dopo la $m$-esima collisione, sceglie un $K$ casuale fra $\{ 0, 1, 2, ..., 2^m - 1 \}$ (binario). Quindi il NIC aspetta per $K \times 512$ tempi bit (per \textit{tempo bit} intendiamo il tempo necessario a trasmettere un bit con bitrate $R$ di mezzo), e quindi torna al passo (2);
			\item Il risultato è che più collisioni si hanno, più lungo è in media l'intervallo di backoff (attesa). 
		\end{itemize}
	Chiaramente, per implementare tale politica dovremo dotarci di un contatore per le $m$ collisioni rilevate. Inoltre, sarà utile prevedere una dimensione massima per la finestra dove scegliere $K$, cioè un numero di collisioni oltre cui gli intervalli di backoff non continuano ad aumentare (in Ethernet questo numero è 10, per cui il tempo massimo è $2^10 -1 = 1023$ tempi bit).
\end{enumerate}

Vediamo cosa abbiamo ottenuto con CSMA su Ethernet: il requisito di completa decentralizzazione è soddisfatto (3), il sistema è effettivamente abbastanza semplice (4), e le frequenze di trasmissione sono perlopià soddisfatte, salvo collisioni da recuperare (requisiti (1) e (2)). Il problema rimasto è quello degli alti carichi: il backoff esponenziale implica che il sistema può rallentare fino a troughput nulli se si verificano collisioni particolarmente gravi (quindi inevitabilmente quando ci sono molti nodi). 

\subsection{MAC a turni}
Veniamo quindi ai protocolli MAC basati sui turni.

\subsubsection{Polling}
Come primo esempio vediamo il meccanismo del \textbf{polling}.
Prevediamo un nodo, detto \textit{master}, che invita gli altri nodi (detti \textit{slave}) a trasmettere a turno.

Questo approccio è usato spesso per dispositivi \textit{"stupidi"} (è usato ad esempio in Bluetooth).

Per quando riguarda i nostri requisiti, non è assolutamente decentralizzato (3), è abbastanza efficiente (requisiti (1) e (2)) ed è sempre abbastanza semplice (4).

Si comporta bene agli alti carichi (il nodo centralizzato governa gli altri assicurando throughput massimo) ma non ai bassi (si sprecano molti turni), proprio come TDMA (sezione 12.3.1).

I problemi sono poi l'\textit{overhead} dato dal polling, la \textit{latenza} introdotta e il fatto che il master rappresenta un \textit{single point of failure}.
Quest'ultimo problema potrebbe essere risolto prevedendo un algoritmo di \textit{rielezione} da mettere in esecuzione al momento della morte del master.
Questo, però, va chiaramente in contro al requisito (4) (semplicità).

\subsubsection{Passaggio di token}
Un altro modo per implementare comunicazioni a turni è il \textbf{token passing}.
Questo token (di controllo) rappresenta un qualche segnalatore che chi vuole trasmettere deve possedere per poterlo fare.
Il token viene passato da un nodo all'altro, sequenzialmente, in modo che tutti i nodi possano parlare.

I problemi principali saranno quindi la \textit{latenza}, la \textit{gestione} del token stesso, nonché il fatto che questo rappresenta nuovamente un \textit{single point of failure} per l'intero sistema (se il token va perso, chi può parlare?).

La latenza si ha dal fatto che il token deve essere passato, e questo rappresenta un \textit{overhead} (il tempo passato a passarsi il token non è passato a trasmettere).



\end{document}
