\documentclass[a4paper,11pt]{article}
\usepackage[a4paper, margin=8em]{geometry}

% usa i pacchetti per la scrittura in italiano
\usepackage[french,italian]{babel}
\usepackage[T1]{fontenc}
\usepackage[utf8]{inputenc}
\frenchspacing 

% usa i pacchetti per la formattazione matematica
\usepackage{amsmath, amssymb, amsthm, amsfonts}

% usa altri pacchetti
\usepackage{gensymb}
\usepackage{hyperref}
\usepackage{standalone}

\usepackage{colortbl}

\usepackage{xstring}
\usepackage{karnaugh-map}

% imposta il titolo
\title{Appunti Reti Informatiche}
\author{Luca Seggiani}
\date{2025}

% imposta lo stile
% usa helvetica
\usepackage[scaled]{helvet}
% usa palatino
\usepackage{palatino}
% usa un font monospazio guardabile
\usepackage{lmodern}

\renewcommand{\rmdefault}{ppl}
\renewcommand{\sfdefault}{phv}
\renewcommand{\ttdefault}{lmtt}

% circuiti
\usepackage{circuitikz}
\usetikzlibrary{babel}

% testo cerchiato
\newcommand*\circled[1]{\tikz[baseline=(char.base)]{
            \node[shape=circle,draw,inner sep=2pt] (char) {#1};}}

% disponi il titolo
\makeatletter
\renewcommand{\maketitle} {
	\begin{center} 
		\begin{minipage}[t]{.8\textwidth}
			\textsf{\huge\bfseries \@title} 
		\end{minipage}%
		\begin{minipage}[t]{.2\textwidth}
			\raggedleft \vspace{-1.65em}
			\textsf{\small \@author} \vfill
			\textsf{\small \@date}
		\end{minipage}
		\par
	\end{center}

	\thispagestyle{empty}
	\pagestyle{fancy}
}
\makeatother

% disponi teoremi
\usepackage{tcolorbox}
\newtcolorbox[auto counter, number within=section]{theorem}[2][]{%
	colback=blue!10, 
	colframe=blue!40!black, 
	sharp corners=northwest,
	fonttitle=\sffamily\bfseries, 
	title=Teorema~\thetcbcounter: #2, 
	#1
}

% disponi definizioni
\newtcolorbox[auto counter, number within=section]{definition}[2][]{%
	colback=red!10,
	colframe=red!40!black,
	sharp corners=northwest,
	fonttitle=\sffamily\bfseries,
	title=Definizione~\thetcbcounter: #2,
	#1
}

% disponi codice
\usepackage{listings}
\usepackage[table]{xcolor}

\definecolor{codegreen}{rgb}{0,0.6,0}
\definecolor{codegray}{rgb}{0.5,0.5,0.5}
\definecolor{codepurple}{rgb}{0.58,0,0.82}
\definecolor{backcolour}{rgb}{0.95,0.95,0.92}

\lstdefinestyle{codestyle}{
		backgroundcolor=\color{black!5}, 
		commentstyle=\color{codegreen},
		keywordstyle=\bfseries\color{magenta},
		numberstyle=\sffamily\tiny\color{black!60},
		stringstyle=\color{green!50!black},
		basicstyle=\ttfamily\footnotesize,
		breakatwhitespace=false,         
		breaklines=true,                 
		captionpos=b,                    
		keepspaces=true,                 
		numbers=left,                    
		numbersep=5pt,                  
		showspaces=false,                
		showstringspaces=false,
		showtabs=false,                  
		tabsize=2
}

\lstdefinestyle{shellstyle}{
		backgroundcolor=\color{black!5}, 
		basicstyle=\ttfamily\footnotesize\color{black}, 
		commentstyle=\color{black}, 
		keywordstyle=\color{black},
		numberstyle=\color{black!5},
		stringstyle=\color{black}, 
		showspaces=false,
		showstringspaces=false, 
		showtabs=false, 
		tabsize=2, 
		numbers=none, 
		breaklines=true
}


\lstdefinelanguage{assembler}{ 
  keywords={AAA, AAD, AAM, AAS, ADC, ADCB, ADCW, ADCL, ADD, ADDB, ADDW, ADDL, AND, ANDB, ANDW, ANDL,
        ARPL, BOUND, BSF, BSFL, BSFW, BSR, BSRL, BSRW, BSWAP, BT, BTC, BTCB, BTCW, BTCL, BTR, 
        BTRB, BTRW, BTRL, BTS, BTSB, BTSW, BTSL, CALL, CBW, CDQ, CLC, CLD, CLI, CLTS, CMC, CMP,
        CMPB, CMPW, CMPL, CMPS, CMPSB, CMPSD, CMPSW, CMPXCHG, CMPXCHGB, CMPXCHGW, CMPXCHGL,
        CMPXCHG8B, CPUID, CWDE, DAA, DAS, DEC, DECB, DECW, DECL, DIV, DIVB, DIVW, DIVL, ENTER,
        HLT, IDIV, IDIVB, IDIVW, IDIVL, IMUL, IMULB, IMULW, IMULL, IN, INB, INW, INL, INC, INCB,
        INCW, INCL, INS, INSB, INSD, INSW, INT, INT3, INTO, INVD, INVLPG, IRET, IRETD, JA, JAE,
        JB, JBE, JC, JCXZ, JE, JECXZ, JG, JGE, JL, JLE, JMP, JNA, JNAE, JNB, JNBE, JNC, JNE, JNG,
        JNGE, JNL, JNLE, JNO, JNP, JNS, JNZ, JO, JP, JPE, JPO, JS, JZ, LAHF, LAR, LCALL, LDS,
        LEA, LEAVE, LES, LFS, LGDT, LGS, LIDT, LMSW, LOCK, LODSB, LODSD, LODSW, LOOP, LOOPE,
        LOOPNE, LSL, LSS, LTR, MOV, MOVB, MOVW, MOVL, MOVSB, MOVSD, MOVSW, MOVSX, MOVSXB,
        MOVSXW, MOVSXL, MOVZX, MOVZXB, MOVZXW, MOVZXL, MUL, MULB, MULW, MULL, NEG, NEGB, NEGW,
        NEGL, NOP, NOT, NOTB, NOTW, NOTL, OR, ORB, ORW, ORL, OUT, OUTB, OUTW, OUTL, OUTSB, OUTSD,
        OUTSW, POP, POPL, POPW, POPB, POPA, POPAD, POPF, POPFD, PUSH, PUSHL, PUSHW, PUSHB, PUSHA, 
				PUSHAD, PUSHF, PUSHFD, RCL, RCLB, RCLW, MOVSL, MOVSB, MOVSW, STOSL, STOSB, STOSW, LODSB, LODSW,
				LODSL, INSB, INSW, INSL, OUTSB, OUTSL, OUTSW
        RCLL, RCR, RCRB, RCRW, RCRL, RDMSR, RDPMC, RDTSC, REP, REPE, REPNE, RET, ROL, ROLB, ROLW,
        ROLL, ROR, RORB, RORW, RORL, SAHF, SAL, SALB, SALW, SALL, SAR, SARB, SARW, SARL, SBB,
        SBBB, SBBW, SBBL, SCASB, SCASD, SCASW, SETA, SETAE, SETB, SETBE, SETC, SETE, SETG, SETGE,
        SETL, SETLE, SETNA, SETNAE, SETNB, SETNBE, SETNC, SETNE, SETNG, SETNGE, SETNL, SETNLE,
        SETNO, SETNP, SETNS, SETNZ, SETO, SETP, SETPE, SETPO, SETS, SETZ, SGDT, SHL, SHLB, SHLW,
        SHLL, SHLD, SHR, SHRB, SHRW, SHRL, SHRD, SIDT, SLDT, SMSW, STC, STD, STI, STOSB, STOSD,
        STOSW, STR, SUB, SUBB, SUBW, SUBL, TEST, TESTB, TESTW, TESTL, VERR, VERW, WAIT, WBINVD,
        XADD, XADDB, XADDW, XADDL, XCHG, XCHGB, XCHGW, XCHGL, XLAT, XLATB, XOR, XORB, XORW, XORL},
  keywordstyle=\color{blue}\bfseries,
  ndkeywordstyle=\color{darkgray}\bfseries,
  identifierstyle=\color{black},
  sensitive=false,
  comment=[l]{\#},
  morecomment=[s]{/*}{*/},
  commentstyle=\color{purple}\ttfamily,
  stringstyle=\color{red}\ttfamily,
  morestring=[b]',
  morestring=[b]"
}

\lstset{language=assembler, style=codestyle}

% disponi sezioni
\usepackage{titlesec}

\titleformat{\section}
	{\sffamily\Large\bfseries} 
	{\thesection}{1em}{} 
\titleformat{\subsection}
	{\sffamily\large\bfseries}   
	{\thesubsection}{1em}{} 
\titleformat{\subsubsection}
	{\sffamily\normalsize\bfseries} 
	{\thesubsubsection}{1em}{}

% tikz
\usepackage{tikz}

% float
\usepackage{float}

% grafici
\usepackage{pgfplots}
\pgfplotsset{width=10cm,compat=1.9}

% disponi alberi
\usepackage{forest}

\forestset{
	rectstyle/.style={
		for tree={rectangle,draw,font=\large\sffamily}
	},
	roundstyle/.style={
		for tree={circle,draw,font=\large}
	}
}

% disponi algoritmi
\usepackage{algorithm}
\usepackage{algorithmic}
\makeatletter
\renewcommand{\ALG@name}{Algoritmo}
\makeatother

% disponi numeri di pagina
\usepackage{fancyhdr}
\fancyhf{} 
\fancyfoot[L]{\sffamily{\thepage}}

\makeatletter
\fancyhead[L]{\raisebox{1ex}[0pt][0pt]{\sffamily{\@title \ \@date}}} 
\fancyhead[R]{\raisebox{1ex}[0pt][0pt]{\sffamily{\@author}}}
\makeatother

\begin{document}

\pagestyle{fancy}
\thispagestyle{empty}
\renewcommand{\thispagestyle}[1]{}

\maketitle
\documentclass[a4paper,11pt]{article}
\usepackage[a4paper, margin=8em]{geometry}

% usa i pacchetti per la scrittura in italiano
\usepackage[french,italian]{babel}
\usepackage[T1]{fontenc}
\usepackage[utf8]{inputenc}
\frenchspacing 

% usa i pacchetti per la formattazione matematica
\usepackage{amsmath, amssymb, amsthm, amsfonts}

% usa altri pacchetti
\usepackage{gensymb}
\usepackage{hyperref}
\usepackage{standalone}

\usepackage{colortbl}

\usepackage{xstring}
\usepackage{karnaugh-map}

% imposta il titolo
\title{Appunti Reti Informatiche}
\author{Luca Seggiani}
\date{2025}

% imposta lo stile
% usa helvetica
\usepackage[scaled]{helvet}
% usa palatino
\usepackage{palatino}
% usa un font monospazio guardabile
\usepackage{lmodern}

\renewcommand{\rmdefault}{ppl}
\renewcommand{\sfdefault}{phv}
\renewcommand{\ttdefault}{lmtt}

% circuiti
\usepackage{circuitikz}
\usetikzlibrary{babel}

% testo cerchiato
\newcommand*\circled[1]{\tikz[baseline=(char.base)]{
            \node[shape=circle,draw,inner sep=2pt] (char) {#1};}}

% disponi il titolo
\makeatletter
\renewcommand{\maketitle} {
	\begin{center} 
		\begin{minipage}[t]{.8\textwidth}
			\textsf{\huge\bfseries \@title} 
		\end{minipage}%
		\begin{minipage}[t]{.2\textwidth}
			\raggedleft \vspace{-1.65em}
			\textsf{\small \@author} \vfill
			\textsf{\small \@date}
		\end{minipage}
		\par
	\end{center}

	\thispagestyle{empty}
	\pagestyle{fancy}
}
\makeatother

% disponi teoremi
\usepackage{tcolorbox}
\newtcolorbox[auto counter, number within=section]{theorem}[2][]{%
	colback=blue!10, 
	colframe=blue!40!black, 
	sharp corners=northwest,
	fonttitle=\sffamily\bfseries, 
	title=Teorema~\thetcbcounter: #2, 
	#1
}

% disponi definizioni
\newtcolorbox[auto counter, number within=section]{definition}[2][]{%
	colback=red!10,
	colframe=red!40!black,
	sharp corners=northwest,
	fonttitle=\sffamily\bfseries,
	title=Definizione~\thetcbcounter: #2,
	#1
}

% disponi codice
\usepackage{listings}
\usepackage[table]{xcolor}

\definecolor{codegreen}{rgb}{0,0.6,0}
\definecolor{codegray}{rgb}{0.5,0.5,0.5}
\definecolor{codepurple}{rgb}{0.58,0,0.82}
\definecolor{backcolour}{rgb}{0.95,0.95,0.92}

\lstdefinestyle{codestyle}{
		backgroundcolor=\color{black!5}, 
		commentstyle=\color{codegreen},
		keywordstyle=\bfseries\color{magenta},
		numberstyle=\sffamily\tiny\color{black!60},
		stringstyle=\color{green!50!black},
		basicstyle=\ttfamily\footnotesize,
		breakatwhitespace=false,         
		breaklines=true,                 
		captionpos=b,                    
		keepspaces=true,                 
		numbers=left,                    
		numbersep=5pt,                  
		showspaces=false,                
		showstringspaces=false,
		showtabs=false,                  
		tabsize=2
}

\lstdefinestyle{shellstyle}{
		backgroundcolor=\color{black!5}, 
		basicstyle=\ttfamily\footnotesize\color{black}, 
		commentstyle=\color{black}, 
		keywordstyle=\color{black},
		numberstyle=\color{black!5},
		stringstyle=\color{black}, 
		showspaces=false,
		showstringspaces=false, 
		showtabs=false, 
		tabsize=2, 
		numbers=none, 
		breaklines=true
}


\lstdefinelanguage{assembler}{ 
  keywords={AAA, AAD, AAM, AAS, ADC, ADCB, ADCW, ADCL, ADD, ADDB, ADDW, ADDL, AND, ANDB, ANDW, ANDL,
        ARPL, BOUND, BSF, BSFL, BSFW, BSR, BSRL, BSRW, BSWAP, BT, BTC, BTCB, BTCW, BTCL, BTR, 
        BTRB, BTRW, BTRL, BTS, BTSB, BTSW, BTSL, CALL, CBW, CDQ, CLC, CLD, CLI, CLTS, CMC, CMP,
        CMPB, CMPW, CMPL, CMPS, CMPSB, CMPSD, CMPSW, CMPXCHG, CMPXCHGB, CMPXCHGW, CMPXCHGL,
        CMPXCHG8B, CPUID, CWDE, DAA, DAS, DEC, DECB, DECW, DECL, DIV, DIVB, DIVW, DIVL, ENTER,
        HLT, IDIV, IDIVB, IDIVW, IDIVL, IMUL, IMULB, IMULW, IMULL, IN, INB, INW, INL, INC, INCB,
        INCW, INCL, INS, INSB, INSD, INSW, INT, INT3, INTO, INVD, INVLPG, IRET, IRETD, JA, JAE,
        JB, JBE, JC, JCXZ, JE, JECXZ, JG, JGE, JL, JLE, JMP, JNA, JNAE, JNB, JNBE, JNC, JNE, JNG,
        JNGE, JNL, JNLE, JNO, JNP, JNS, JNZ, JO, JP, JPE, JPO, JS, JZ, LAHF, LAR, LCALL, LDS,
        LEA, LEAVE, LES, LFS, LGDT, LGS, LIDT, LMSW, LOCK, LODSB, LODSD, LODSW, LOOP, LOOPE,
        LOOPNE, LSL, LSS, LTR, MOV, MOVB, MOVW, MOVL, MOVSB, MOVSD, MOVSW, MOVSX, MOVSXB,
        MOVSXW, MOVSXL, MOVZX, MOVZXB, MOVZXW, MOVZXL, MUL, MULB, MULW, MULL, NEG, NEGB, NEGW,
        NEGL, NOP, NOT, NOTB, NOTW, NOTL, OR, ORB, ORW, ORL, OUT, OUTB, OUTW, OUTL, OUTSB, OUTSD,
        OUTSW, POP, POPL, POPW, POPB, POPA, POPAD, POPF, POPFD, PUSH, PUSHL, PUSHW, PUSHB, PUSHA, 
				PUSHAD, PUSHF, PUSHFD, RCL, RCLB, RCLW, MOVSL, MOVSB, MOVSW, STOSL, STOSB, STOSW, LODSB, LODSW,
				LODSL, INSB, INSW, INSL, OUTSB, OUTSL, OUTSW
        RCLL, RCR, RCRB, RCRW, RCRL, RDMSR, RDPMC, RDTSC, REP, REPE, REPNE, RET, ROL, ROLB, ROLW,
        ROLL, ROR, RORB, RORW, RORL, SAHF, SAL, SALB, SALW, SALL, SAR, SARB, SARW, SARL, SBB,
        SBBB, SBBW, SBBL, SCASB, SCASD, SCASW, SETA, SETAE, SETB, SETBE, SETC, SETE, SETG, SETGE,
        SETL, SETLE, SETNA, SETNAE, SETNB, SETNBE, SETNC, SETNE, SETNG, SETNGE, SETNL, SETNLE,
        SETNO, SETNP, SETNS, SETNZ, SETO, SETP, SETPE, SETPO, SETS, SETZ, SGDT, SHL, SHLB, SHLW,
        SHLL, SHLD, SHR, SHRB, SHRW, SHRL, SHRD, SIDT, SLDT, SMSW, STC, STD, STI, STOSB, STOSD,
        STOSW, STR, SUB, SUBB, SUBW, SUBL, TEST, TESTB, TESTW, TESTL, VERR, VERW, WAIT, WBINVD,
        XADD, XADDB, XADDW, XADDL, XCHG, XCHGB, XCHGW, XCHGL, XLAT, XLATB, XOR, XORB, XORW, XORL},
  keywordstyle=\color{blue}\bfseries,
  ndkeywordstyle=\color{darkgray}\bfseries,
  identifierstyle=\color{black},
  sensitive=false,
  comment=[l]{\#},
  morecomment=[s]{/*}{*/},
  commentstyle=\color{purple}\ttfamily,
  stringstyle=\color{red}\ttfamily,
  morestring=[b]',
  morestring=[b]"
}

\lstset{language=assembler, style=codestyle}

% disponi sezioni
\usepackage{titlesec}

\titleformat{\section}
	{\sffamily\Large\bfseries} 
	{\thesection}{1em}{} 
\titleformat{\subsection}
	{\sffamily\large\bfseries}   
	{\thesubsection}{1em}{} 
\titleformat{\subsubsection}
	{\sffamily\normalsize\bfseries} 
	{\thesubsubsection}{1em}{}

% tikz
\usepackage{tikz}

% float
\usepackage{float}

% grafici
\usepackage{pgfplots}
\pgfplotsset{width=10cm,compat=1.9}

% disponi alberi
\usepackage{forest}

\forestset{
	rectstyle/.style={
		for tree={rectangle,draw,font=\large\sffamily}
	},
	roundstyle/.style={
		for tree={circle,draw,font=\large}
	}
}

% disponi algoritmi
\usepackage{algorithm}
\usepackage{algorithmic}
\makeatletter
\renewcommand{\ALG@name}{Algoritmo}
\makeatother

% disponi numeri di pagina
\usepackage{fancyhdr}
\fancyhf{} 
\fancyfoot[L]{\sffamily{\thepage}}

\makeatletter
\fancyhead[L]{\raisebox{1ex}[0pt][0pt]{\sffamily{\@title \ \@date}}} 
\fancyhead[R]{\raisebox{1ex}[0pt][0pt]{\sffamily{\@author}}}
\makeatother

\begin{document}
% sezione (data)
\section{Lezione del 23-09-25}

% stili pagina
\thispagestyle{empty}
\pagestyle{fancy}

% testo
\subsection{Introduzione}
Il corso si pone di presentare le nozioni di base sulle reti informatiche, le tecnologie di rete più diffuse, i protocolli Internet e lo sviluppo di applicazioni distribuite \textit{client-server} e \textit{peer-to-peer} (\textbf{P2P}).

In particolare il programma del corso comprende:
\begin{itemize}
	\item Sviluppo di \textbf{applicazioni} in rete:
		\begin{itemize}
			\item Client-server;
			\item Peer-to-peer.
		\end{itemize}
	\item Reti a \textbf{connessione diretta}:
		\begin{itemize}
			\item Collegamenti punto-punto;
			\item Reti locali.
		\end{itemize}
	\item Reti a \textbf{commutazione di pacchetto};
	\item \textbf{Interconnessione} di reti di tipo diverso;
	\item \textbf{Trasporto} end-to-end e protocolli;
	\item \textbf{Sicurezza};
	\item Reti \textbf{wireless} e \textbf{mobili}, intese come caso particolare delle normali reti \textbf{cablate} (\textit{wired}).
\end{itemize}

\subsubsection{Applicazioni in rete}
Nel dettaglio delle \textit{applicazioni in rete}, vedremo come già detto i paradigmi \textit{client-server} e \textit{peer-to-peer}, di cui possiamo già fare alcuni esempi:
\begin{itemize}
	\item Applicazioni client-server:
		\begin{itemize}
			\item Web;
			\item Trasferimento file;
			\item Posta elettronica;
			\item DNS;
			\item Ecc...
		\end{itemize}
	\item Applicazioni peer-to-peer:
		\begin{itemize}
			\item Ricerca di contenuti;
			\item Torrent;
			\item Telefonia online;
			\item Ecc...
		\end{itemize}
\end{itemize}

In questo ci avvarremo del concetto di \textbf{socket} come primitiva per la gestione della rete dal lato S/O.

\subsubsection{Reti dirette, a commutazione e wireless}
Inizieremo con lo studio di \textit{collegamenti punto-punto}, e quindi di trasferimento affidabile di dati fra 2 punti.
Vedremo poi le reti locali, ad accesso multiplo, e i casi particolari come \textit{Ethernet}.

Vedremo quindi le reti a \textit{commutazione di pacchetto} per la copertura di grandi regioni. Anche qui approfondiremo tecnologie come gli \textit{switch}, ancora \textit{Ethernet}, ecc...

Per quanto riguarda l'\textit{interconnessione di reti} vedremo il protocollo Internet \textbf{IPv4}, il \textbf{routing} (cioè l'\textit{instradamento}) e i protocolli di trasporto (\textbf{UDP} e \textbf{TCP}).

Parleremo anche di reti \textit{wireless} e \textit{mobili}, e quindi di tecnologie come \textbf{WiFi}, le \textbf{reti cellulari}, e reti senza infrastruttura come \textbf{Bluetooth}.

\subsubsection{Sicurezza}
Vedremo poi le minacce alla \textit{sicurezza} e alcune soluzioni che abbiamo a disposizione per mitigarle.
In particolare, tratteremo di \textbf{crittografia} e \textbf{integrità} dei messaggi. 

Nello specifico parleremo di tecnologie a livello applicazione (\textbf{PGP}), a livello trasporto (\textbf{TLS} (usata in \textit{HTTPS})), a livello Internet (\textbf{IP-Sec}) e difese di sicurezza come \textbf{firewall} e \textbf{IDS}.

\subsection{Terminologia}
Iniziamo quindi a definire la terminologia di base usata nel corso, usando Internet come esempio.

\subsubsection{Internet}
La prima domanda che ci poniamo è \textit{"Che cos'è Internet?"}.

\par\medskip
\textbf{\textsf{Visione ingegneristica}} \\
\noindent
Iniziamo col vedere la definizione di Internet agli occhi di un ingegnere che si occupa di reti: 
\begin{itemize}
	\item Si tratterà di una rete che connette miliardi di \textit{dispositivi}, detti \textbf{host} (\textit{"ospiti"}), che eseguono \textit{applicazioni in rete} al cosiddetto \textbf{edge} (\textit{"bordo"}) della rete.
	\item Una visione \textbf{interna} della rete ci dirà invece che è un insieme di \textbf{pacchetti} che viaggiano attraverso infrastruttura (\textit{router}, \textit{switch}), per raggiungere il loro destinatario.
	\item A livello \textbf{fisico} potremmo considerare le connessioni fisiche fra dispositivi, date da cavi, segnali radio, ecc...
	\item Infine, potremo organizzare le \textbf{reti} come collezioni di dispositivi, router e connessioni gestite da determinate organizzazioni.
\end{itemize}

Non è esattamente corretto parlare di \textit{"reti di calcolatori"} in quanto oggi ad essere connessi a Internet sono tutta una gamma di dispositivi non necessariamente orientati al puro \textit{calcolo}: è questo il caso del cosiddetto \textit{Internet of Things} (\textbf{IoT}).

\par\medskip

Possiamo quindi intendere Internet come una "rete di reti", cioè più \textbf{ISP} (\textit{Internet Service Providers}) connessi fra di loro, che a loro volta connettono una gamma dispositivi (host, router, switch, ecc...).

Per governare l'operazione di tali rete si necessita di \textbf{protocolli}, che definiscono il modo in cui si inviano e ricevono messaggi in rete.

In particolare, per quanto riguarda Internet notiamo l'\textbf{IETF} (\textit{Internet Engineering Task Force}), organizzazione che gestisce diversi standard del settore (anche detti \textbf{RFC}, da \textit{Request For Comments memoranda}).

\par\medskip
\textbf{\textsf{Visione utente}} \\
\noindent
Per l'utente, internet sarà un insieme di \textbf{infrastrutture} che forniscono \textbf{servizi} finali, fra cui il Web, telecomunicazioni, streaming, ecc...
Dal punto di vista delle \textbf{applicazioni} in esecuzione sui dispositivi, Internet rappresenterà un'interfaccia di programmazione per consentire la comunicazione fra processi su una o più macchine.
In questo parleremo di \textbf{hook} che permettono alle applicazioni di \textbf{connettersi} a Internet, cioè accedere ad un qualche protocollo di trasporto dei dati. 

\subsubsection{Protocolli}
Un \textbf{protocollo} è una precisa \textit{specifica} del formato secondo il quale due dispositivi in rete si scambiano informazioni.
Solitamente i protocolli si sviluppano in più fasi, successive nel tempo, dove si portano avanti diverse operazioni necessarie alla comunicazione.

Esempi di protocollo sono il \textit{protocollo Internet} \textbf{IPv4}, e il \textit{protocollo di trasporto} \textbf{TCP} usato nel \textit{Web} e visto nel corso di progettazione web.

\subsubsection{Infrastruttura di Internet}
Vediamo più nel dettaglio la struttura di Internet:
\begin{itemize}
	\item 
	Abbiamo detto che all'\textit{edge} di internet ci sono i cosiddetti \textit{host}, cioè i \textbf{client} e i \textbf{server}. Notiamo che non vogliamo riferirci alle macchine fisiche client o server, ma ai \textbf{processi} che si comportano come tali per l'implementazione di un'applicazione distribuita.
	\item
	I dispositivi \textit{terminali} che abbiamo appena nominato accedono ad Internet attraverso le cosiddette \textbf{reti di accesso}, cablate o wireless e basate sulle tecnologie utilizzate (router).

	Per collegare i sistemi terminali ai router si usano \textit{reti residenziali}, \textit{reti di accesso istituzionali} (scuola, lavoro, ecc...), nonché \textit{reti wireless e mobili} (Wifi, o reti come 4G solitamente fornite da privati).
	In questo caso può interessarci la frequenza di trasmissione, in bit al secondo, di una rete di accesso, o se quella rete è ad accesso \textit{condiviso} (pensa WiFi) o \textit{dedicato} (pensa Ethernet).

	Uno standard storico per le reti di accesso è quello della trasmissione sulla linea telefonica su \textbf{DSL} (\textit{Digital Subscriber Line}).
	Negli Stati Uniti si è invece diffuso l'uso dela linea televisiva cablata.
	Oggi, sfruttiamo invece tecnologie come \textbf{ADSL} (\textit{Asymmetric Digital Subscriber Line}) e \textbf{FTTC} (\textit{Fiber To The Cabinet}).
	La differenza principale fra queste è che la linea in ADSL è interamente in rame, sia dalla centrale all'armadio di ripartizione che dall'armadio ripartilinea agli utenti finali, mentre nella linea FTTC si porta il segnale all'armadio attraverso cavi in fibra ottica.
	Lo standard di ultima generazione è \textbf{FTTH} (\textit{Fiber To The Home}), che prevede una linea in fibra ottica anche dall'armadio agli utenti finali.

	Possiamo quindi vedere la rete locale (\textbf{LAN} (\textit{Local Area Network})) di una comune abitazione come composta da un router, connesso a un \textbf{modem} DSL (\textit{modem} deriva da modulatore/demodulatore sulla linea telefonica dei messaggi Internet) o direttamente via cavo ad un altro centro di ripartizione, e ad eventuali dispositivi come \textit{access point WiFi} che offrono la connessione via rete mobile ai dispositivi finali (una cosiddetta \textbf{WLAN}, \textit{Wireless Local Area Network}).

	Altre soluzioni per le comunicazioni wireless sono rappresentati da reti \textbf{cellulari} su larga scala, che sono quelle usate dagli operatori telefonici (tecnologie come \textbf{4G}, ecc...).

	\item 
	Dalle reti di accesso si arriva a Internet attraverso reti interconnesse di router, arrivando quindi alle \textit{reti di reti} di cui stavamo parlando.
\end{itemize}


\end{document}

\documentclass[a4paper,11pt]{article}
\usepackage[a4paper, margin=8em]{geometry}

% usa i pacchetti per la scrittura in italiano
\usepackage[french,italian]{babel}
\usepackage[T1]{fontenc}
\usepackage[utf8]{inputenc}
\frenchspacing 

% usa i pacchetti per la formattazione matematica
\usepackage{amsmath, amssymb, amsthm, amsfonts}

% usa altri pacchetti
\usepackage{gensymb}
\usepackage{hyperref}
\usepackage{standalone}

\usepackage{colortbl}

\usepackage{xstring}
\usepackage{karnaugh-map}

% imposta il titolo
\title{Appunti Reti Informatiche}
\author{Luca Seggiani}
\date{2025}

% imposta lo stile
% usa helvetica
\usepackage[scaled]{helvet}
% usa palatino
\usepackage{palatino}
% usa un font monospazio guardabile
\usepackage{lmodern}

\renewcommand{\rmdefault}{ppl}
\renewcommand{\sfdefault}{phv}
\renewcommand{\ttdefault}{lmtt}

% circuiti
\usepackage{circuitikz}
\usetikzlibrary{babel}

% testo cerchiato
\newcommand*\circled[1]{\tikz[baseline=(char.base)]{
            \node[shape=circle,draw,inner sep=2pt] (char) {#1};}}

% disponi il titolo
\makeatletter
\renewcommand{\maketitle} {
	\begin{center} 
		\begin{minipage}[t]{.8\textwidth}
			\textsf{\huge\bfseries \@title} 
		\end{minipage}%
		\begin{minipage}[t]{.2\textwidth}
			\raggedleft \vspace{-1.65em}
			\textsf{\small \@author} \vfill
			\textsf{\small \@date}
		\end{minipage}
		\par
	\end{center}

	\thispagestyle{empty}
	\pagestyle{fancy}
}
\makeatother

% disponi teoremi
\usepackage{tcolorbox}
\newtcolorbox[auto counter, number within=section]{theorem}[2][]{%
	colback=blue!10, 
	colframe=blue!40!black, 
	sharp corners=northwest,
	fonttitle=\sffamily\bfseries, 
	title=Teorema~\thetcbcounter: #2, 
	#1
}

% disponi definizioni
\newtcolorbox[auto counter, number within=section]{definition}[2][]{%
	colback=red!10,
	colframe=red!40!black,
	sharp corners=northwest,
	fonttitle=\sffamily\bfseries,
	title=Definizione~\thetcbcounter: #2,
	#1
}

% disponi codice
\usepackage{listings}
\usepackage[table]{xcolor}

\definecolor{codegreen}{rgb}{0,0.6,0}
\definecolor{codegray}{rgb}{0.5,0.5,0.5}
\definecolor{codepurple}{rgb}{0.58,0,0.82}
\definecolor{backcolour}{rgb}{0.95,0.95,0.92}

\lstdefinestyle{codestyle}{
		backgroundcolor=\color{black!5}, 
		commentstyle=\color{codegreen},
		keywordstyle=\bfseries\color{magenta},
		numberstyle=\sffamily\tiny\color{black!60},
		stringstyle=\color{green!50!black},
		basicstyle=\ttfamily\footnotesize,
		breakatwhitespace=false,         
		breaklines=true,                 
		captionpos=b,                    
		keepspaces=true,                 
		numbers=left,                    
		numbersep=5pt,                  
		showspaces=false,                
		showstringspaces=false,
		showtabs=false,                  
		tabsize=2
}

\lstdefinestyle{shellstyle}{
		backgroundcolor=\color{black!5}, 
		basicstyle=\ttfamily\footnotesize\color{black}, 
		commentstyle=\color{black}, 
		keywordstyle=\color{black},
		numberstyle=\color{black!5},
		stringstyle=\color{black}, 
		showspaces=false,
		showstringspaces=false, 
		showtabs=false, 
		tabsize=2, 
		numbers=none, 
		breaklines=true
}


\lstdefinelanguage{assembler}{ 
  keywords={AAA, AAD, AAM, AAS, ADC, ADCB, ADCW, ADCL, ADD, ADDB, ADDW, ADDL, AND, ANDB, ANDW, ANDL,
        ARPL, BOUND, BSF, BSFL, BSFW, BSR, BSRL, BSRW, BSWAP, BT, BTC, BTCB, BTCW, BTCL, BTR, 
        BTRB, BTRW, BTRL, BTS, BTSB, BTSW, BTSL, CALL, CBW, CDQ, CLC, CLD, CLI, CLTS, CMC, CMP,
        CMPB, CMPW, CMPL, CMPS, CMPSB, CMPSD, CMPSW, CMPXCHG, CMPXCHGB, CMPXCHGW, CMPXCHGL,
        CMPXCHG8B, CPUID, CWDE, DAA, DAS, DEC, DECB, DECW, DECL, DIV, DIVB, DIVW, DIVL, ENTER,
        HLT, IDIV, IDIVB, IDIVW, IDIVL, IMUL, IMULB, IMULW, IMULL, IN, INB, INW, INL, INC, INCB,
        INCW, INCL, INS, INSB, INSD, INSW, INT, INT3, INTO, INVD, INVLPG, IRET, IRETD, JA, JAE,
        JB, JBE, JC, JCXZ, JE, JECXZ, JG, JGE, JL, JLE, JMP, JNA, JNAE, JNB, JNBE, JNC, JNE, JNG,
        JNGE, JNL, JNLE, JNO, JNP, JNS, JNZ, JO, JP, JPE, JPO, JS, JZ, LAHF, LAR, LCALL, LDS,
        LEA, LEAVE, LES, LFS, LGDT, LGS, LIDT, LMSW, LOCK, LODSB, LODSD, LODSW, LOOP, LOOPE,
        LOOPNE, LSL, LSS, LTR, MOV, MOVB, MOVW, MOVL, MOVSB, MOVSD, MOVSW, MOVSX, MOVSXB,
        MOVSXW, MOVSXL, MOVZX, MOVZXB, MOVZXW, MOVZXL, MUL, MULB, MULW, MULL, NEG, NEGB, NEGW,
        NEGL, NOP, NOT, NOTB, NOTW, NOTL, OR, ORB, ORW, ORL, OUT, OUTB, OUTW, OUTL, OUTSB, OUTSD,
        OUTSW, POP, POPL, POPW, POPB, POPA, POPAD, POPF, POPFD, PUSH, PUSHL, PUSHW, PUSHB, PUSHA, 
				PUSHAD, PUSHF, PUSHFD, RCL, RCLB, RCLW, MOVSL, MOVSB, MOVSW, STOSL, STOSB, STOSW, LODSB, LODSW,
				LODSL, INSB, INSW, INSL, OUTSB, OUTSL, OUTSW
        RCLL, RCR, RCRB, RCRW, RCRL, RDMSR, RDPMC, RDTSC, REP, REPE, REPNE, RET, ROL, ROLB, ROLW,
        ROLL, ROR, RORB, RORW, RORL, SAHF, SAL, SALB, SALW, SALL, SAR, SARB, SARW, SARL, SBB,
        SBBB, SBBW, SBBL, SCASB, SCASD, SCASW, SETA, SETAE, SETB, SETBE, SETC, SETE, SETG, SETGE,
        SETL, SETLE, SETNA, SETNAE, SETNB, SETNBE, SETNC, SETNE, SETNG, SETNGE, SETNL, SETNLE,
        SETNO, SETNP, SETNS, SETNZ, SETO, SETP, SETPE, SETPO, SETS, SETZ, SGDT, SHL, SHLB, SHLW,
        SHLL, SHLD, SHR, SHRB, SHRW, SHRL, SHRD, SIDT, SLDT, SMSW, STC, STD, STI, STOSB, STOSD,
        STOSW, STR, SUB, SUBB, SUBW, SUBL, TEST, TESTB, TESTW, TESTL, VERR, VERW, WAIT, WBINVD,
        XADD, XADDB, XADDW, XADDL, XCHG, XCHGB, XCHGW, XCHGL, XLAT, XLATB, XOR, XORB, XORW, XORL},
  keywordstyle=\color{blue}\bfseries,
  ndkeywordstyle=\color{darkgray}\bfseries,
  identifierstyle=\color{black},
  sensitive=false,
  comment=[l]{\#},
  morecomment=[s]{/*}{*/},
  commentstyle=\color{purple}\ttfamily,
  stringstyle=\color{red}\ttfamily,
  morestring=[b]',
  morestring=[b]"
}

\lstset{language=assembler, style=codestyle}

% disponi sezioni
\usepackage{titlesec}

\titleformat{\section}
	{\sffamily\Large\bfseries} 
	{\thesection}{1em}{} 
\titleformat{\subsection}
	{\sffamily\large\bfseries}   
	{\thesubsection}{1em}{} 
\titleformat{\subsubsection}
	{\sffamily\normalsize\bfseries} 
	{\thesubsubsection}{1em}{}

% tikz
\usepackage{tikz}

% float
\usepackage{float}

% grafici
\usepackage{pgfplots}
\pgfplotsset{width=10cm,compat=1.9}

% disponi alberi
\usepackage{forest}

\forestset{
	rectstyle/.style={
		for tree={rectangle,draw,font=\large\sffamily}
	},
	roundstyle/.style={
		for tree={circle,draw,font=\large}
	}
}

% disponi algoritmi
\usepackage{algorithm}
\usepackage{algorithmic}
\makeatletter
\renewcommand{\ALG@name}{Algoritmo}
\makeatother

% disponi numeri di pagina
\usepackage{fancyhdr}
\fancyhf{} 
\fancyfoot[L]{\sffamily{\thepage}}

\makeatletter
\fancyhead[L]{\raisebox{1ex}[0pt][0pt]{\sffamily{\@title \ \@date}}} 
\fancyhead[R]{\raisebox{1ex}[0pt][0pt]{\sffamily{\@author}}}
\makeatother

\begin{document}
% sezione (data)
\section{Lezione del 24-09-25}

% stili pagina
\thispagestyle{empty}
\pagestyle{fancy}

% testo
\subsection{Comunicazione dati su Internet}
Abbiamo visto la struttura a livello fisico della rete Internet.
Vediamo adesso i meccanismi secondo cui la trasmissione di dati avviene.
La rete Internet è una rete a \textbf{commutazione di pacchetto}, il compito degli host è di:
\begin{itemize}
	\item Ottenere messaggi dalle applicazioni;
	\item Dividere quei messaggi in frammenti più piccoli, detti \textbf{pacchetti}, di dimensione $L$ bit;
	\item Trasmettere quei pacchetti nella rete di accesso ad una \textit{frequenza di trasmissione} (o \textbf{bit-rate}) $R$.
\end{itemize}

Il tempo $T_{\text{packet}}$ necessario a trasmettere un pacchetto da $L$ bit su una linea da $R$ bit al secondo di bitrate sarà quindi semplicemente calcolato come:
$$
T_{\text{packet}} = \frac{L}{R}
$$

Il bitrate, detto anche frequenza di \textit{link}, dipende appunto dal \textbf{link} (o \textit{mezzo}) della trasmissione, cioè l'infrastruttura fisica che sta fra trasmettitore e ricevitore.
Possiamo classificare 2 tipi di link:
\begin{itemize}
	\item Mezzi \textbf{guidati}: segnali che si propagano in mezzi solidi (rame, fibra, cavi coassiali, ecc...). 

		\begin{itemize}
			\item 
		Un celebre esempio di mezzo trasmissivo guidato è il classico \textbf{doppino telefonico}, o trasmettitore a \textit{Twisted Pair} (\textbf{TP}). Questo è formato da due fili di rame isolati e avvolti l'uno sull'altro, che permettono la trasmissione differenziale e quindi la riduzione dei rumori in \textit{common-mode};
			\item
		Un altro tipo di mezzo trasmissivo guidato è il \textbf{cavo coassiale}, formato da due conduttori concentrici in rame separati da un dielettrico. Il segnale è trasferito come campo magnetico fra i due conduttori: questo permette bitrate più alti rispetto al normale doppino telefonico e una migliore schermatura dalle interferenze;
			\item
		Infine, possiamo parlare della \textbf{fibra ottica}, formata da fibra di vetro che porta impulsi luminosi ad altissima velocità. Questa tecnologia presenta velocità di trasmissione estremamente alte, e vista la natura luminosa del segnale, non è suscettibile ad interferenze (alta affidabilità, cioè piccola frequenza di errore sui bit).
		\end{itemize}

	\item Mezzi \textbf{non guidati}: segnali che si propagano nell'etere (segnali radio, ecc...). Questi sono solitamente meno sicuri ma significativamente più comodi per l'utente finale (possibilità di spostarsi, mancanza di cavi, ecc...).

		La trasmissione wireless è suscettibile a fenomeni fisici come \textit{riflessi}, \textit{interferenze} e \textit{ostruzione} da parte di oggetti fisici.

		\begin{itemize}
			\item Il \textbf{WiFi} è un esempio di mezzo di trasmissione non guidato che può raggiungere centinaia di Mbps su regioni locali;
			\item Reti wireless più ampie possono essere quelle \textbf{cellulari}, usate nella telefonia mobile;
			\item Infine si può parlare delle \textbf{reti satellitari}, usate per l'interconnessione di regioni geografiche fra di loro anche molto distanti.
		\end{itemize}

\end{itemize}

\subsubsection{Commutazione di circuito}
Prima della commutazione di pacchetto si usava la tecnica della \textbf{commutazione di circuito} (ad esempio sulle linee telefoniche).
Questo prevede di dedicare completamente una certa linea di trasmissione alla comunicazione fra due host, invalidandone quindi l'uso da parte di altri host.

Chiaramente, la commutazione di pacchetto permette un carico migliore della linea, dove più pacchetti provenienti da diverse fonti possono viaggiare a istanti temporali molto vicini fra di loro.

In particolare, la commutazione di paccheto è utile per dati trasmessi in \textit{burst}, mentre la rete a commutazione di circuito assicura minima congestione possibile a costo di occupazione completa della linea.

\subsubsection{Commutazione di pacchetto}
La tecnica della \textbf{commutazione di pacchetto} o \textit{packet-switching} permette ad una rete di \textbf{router} interconnessi di ricevere ed instradare (letteralmente, \textit{"routing"}) pacchetti provenienti da più fonti in modo che raggiungano la loro destinazione.

Questo inserisce chiaramente un ritardo nel sistema, in quanto il router deve:
\begin{itemize}
	\item Ricevere il pacchetto \textit{completamente} ed memorizzarlo: questo richiede $L/R$ secondi;
	\item Leggere l'header del pacchetto per capire il prossimo passo dell'instradamento;
	\item Trasmettere il pacchetto verso la sua nuova destinazione (un altro router o l'host finale), impeigando ancora $L/R$ secondi.
\end{itemize}

Abbiamo quindi che il ritardo end-end immesso dal router è necessariamente di almeno $2L/R$ secondi, tralasciando il tempo necessario all'instradamento stesso. 

Nel caso generale si abbiano $N$ router (quindi $N + 1$ link fra i router) e $P$ pacchetti da inviare, dovremo considerareche il primo pacchetto arriva in $(N + 1) \frac{L}{R}$ (deve attraversare tutti i link) e i successivi $P - 1$ pacchetti arrivano in $(P - 1) \frac{L}{R}$, per cui il tempo complessivo è:
$$
T_{end-to-end} = (N + P) \frac{L}{R}
$$

Se troppi pacchetti arrivano in un breve lasso di tempo, cioè se la frequenza di arrivo supera quella di trasmissione:
\begin{itemize}
	\item I pacchetti verrano messi in coda finché non sarà possibile trasmetterli;
	\item I pacchetti possono essere persi se il buffer di memoria dedicato alla loro memorizzazione nel ruoter si riempie.
\end{itemize}

\subsection{Prestazioni della commutazione di pacchetto}
Facciamo qualche considerazione ulteriore sulle prestazioni delle linee a commutazione di pacchetto.
Abbiamo ottenuto il valore $T_{\text{packet}} = \frac{L}{R}$ per la trasmissione di un singolo pacchetto da $L$ bit su una linea con bitrate $R$, e $T_{\text{end-to-end}} = (N + P) \frac{L}{R}$ per più pacchetti su un numero arbitrario di router.

Da quanto abbiamo detto nella scorsa sezione, un modello più sofisticato del packet-switching terrà conto di 4 sorgenti di ritardo:
\begin{itemize}
	\item Ritardo di \textbf{trasmissione} $T_{\text{trans}}$, dato dalle caratteristiche del link. Come abbiamo già detto, questo vale:
		$$
		T_{\text{trans}} = \frac{L}{R}
		$$
		con $L$ lunghezza del pacchetto in bit e $R$ bitrate del link;
	\item Ritardo di \textbf{propagazione} $T_{\text{prop}}$, dato dalle proprietà fisiche del mezzo di trasmissione. In particolare, questo è il tempo fisico di trasmissione del segnale su un link, dato da:
		$$
		T_{\text{prop}} = \frac{d}{s}
		$$
con $d$ distanza del link e $s$ velocità del mezzo di trasmissione. Chiaramente, per i nostri scopi $s$ sara una frazione significativa della velocità della luce $c \approx 3 \cdot 10^8$;
	\item Ritardo di \textbf{laborazione} (instradamento) $T_{\text{proc}}$, dipende dalle caratteristiche del router ed è perlopiù costante;
	\item Ritardo di \textbf{accodamento} dato dalla presenza di code $T_{\text{queue}}$. Questo è il più complicato da trattare, in quanto dipende dal numero di pacchetti presenti nel buffer del router. Come vedremo fra poco, una buona euristica per la valutazione di questo ritardo (che è comunque trattabile solo in maniera statistica) è l'\textit{intensità di traffico} sulla linea di trasmissione.
\end{itemize}
Sommando queste sorgenti di ritardo potremo ottenere una stima del ritardo complessivo su un router (nodo) $T_{\text{node}}$:
$$
T_{\text{node}} = T_{\text{trans}} + T_{\text{prop}} + T_{\text{proc}} + T_{\text{queue}}
$$

Dati $N$ nodi, il ritardo end-to-end potrà quindi essere calcolato semplicemente come:
$$
T_{\text{end-to-end}} = N T_{\text{node}}
$$

\subsubsection{Ritardo di accodamento}
Come anticipatp, possiamo valutare statisticamente il ritardo di accodamento $T_{\text{queue}}$ calcolando l'\textit{intensità di traffico} su una linea:
$$
I_{\text{traffic}} = \frac{L \alpha}{R}
$$
presa $\alpha$ come la frequenza media di trasmissione di pacchetti sull'istante temporale.

Semplicemente tracciando la funzione si nota che se $I_{\text{traffic}} >= 1$ il tempo di di accodamento tende a infinito (si ha necessariamente perdita di pacchetti), per cui vorremo mantenere $I_{\text{traffic}} < 1$, e idealmente $I_{\text{traffic}} << 1$.

\subsubsection{Traceroute}
Per fare diagnostica in situazioni reali, si possono usare software di \textbf{traceroute}, che inviano pacchetti con lo scopo di tracciare gli host incontrati e misurare il tempo impiegato nella trasmissione \textit{round-trip}, o almeno capire se i pacchetti sono stati inviati o persi. 

\subsubsection{Troughput}
Introduciamo il concetto di \textbf{troughput} come la frequenza (bit al secondo) con cui i bit vengono spediti da trasmettitore a ricevitore.
Il throughput può essere:
\begin{itemize}
	\item \textbf{Istantaneo}: calcolato ad un certo istante temporale (per quanto possibile dalla natura discreta della trasmissione);
	\item \textbf{Medio}: calcolato su un periodo temporale più lungo.
\end{itemize}

Ipotizziamo che un server debba inviare un file di $F$ bit ad un client.
La linea di comunicazione fra server e router ha capacità di link di $R_S$ bit/secondo, mentre la linea fra router e client ha capacità di $R_C$ bit/secondo.
Chiaramente la capacità totale sarà:
$$
R_F = \min(R_S, R_C)
$$
cioè link con basse capacità di trasmissione rappresentano \textit{bottleneck} per il throughput di tutto il collegamento fra dispositivi.

\par\smallskip

Facciamo adesso l'esempio di una rete condivisa fra più coppie client/server.
In questo caso, date $N$ coppie, la capacità della rete di interconnessione agli occhi di una singola coppia sarà, preso $R$ come la capacità complessiva:
$$
R_i = \frac{R}{N}
$$
per cui ogni coppia vedrà una linea di comunicazione con capacità:
$$
R_F = \min(R_S, R_C, \frac{R}{N})
$$

\subsection{Struttura di Internet}
Abbiamo detto che Internet è una rete di reti.
\begin{itemize}
	\item Gli host si collegano a internet attraverso le \textit{reti di accesso} (reti LAN, istituzionali, reti mobili, ecc...);
	\item Le reti di accesso si collegano agli \textbf{ISP} (\textit{Internet Server Provider});
	\item Gli ISP devono essere collegati ad altri ISP per permettere la comunicazione fra host su diversi ISP (all'interno di \textbf{IXP} (\textit{Internet eXchange Point}, ecc...)).
\end{itemize}

La struttura di reti interconnesse che si viene a formare è molto complessa, e la sua evoluzione è stata guidata da fattori economici e politici.

Abbiamo quindi una struttura gerarchica:
\begin{enumerate}
	\item Al livello più alto troviamo i cosiddetti \textbf{tier 1 ISP} e i \textbf{content provider} (Google, Facebook, ecc...) che si occupano di copertura nazionale e internazionale. In particolare, i content provider preferiscono collegarsi direttamente agli IXP per risparmiare sugli ISP;
	\item Seguono gli \textbf{IXP}, che collegano più ISP fra di loro, e gli ISP locali (regionali, ecc...);
	\item Infine troviamo le reti di 	\textbf{accesso} locale (reti LAN, WLAN, ecc...).
\end{enumerate}

\subsection{Sicurezza Internet}
La struttura di Internet espone i suoi utenti a diversi pericoli, fra cui:
\begin{itemize}
	\item \textbf{Virus}: programmi maligni che si replicano modificando altri programmi;
	\item \textbf{Worm}: programmi maligni che si replicano con lo scopo di diffondersi in altri computer;
	\item \textbf{Spyware}: programmi maligni che cercano di ottenere informazioni che violano la privacy;
\end{itemize}
ed altre svariate categorie di \textit{malware}.

\subsubsection{Attacchi DoS}
Un caso tipico di attacchi informatici in rete è quello degli attacchi \textbf{DoS} (\textit{Denial of Service}), dove l'attaccante cerca di sfruttare le risorse di un server (memoria, larghezza di banda) al punto di renderlo inutilizzabile ad altri utente.
Questo può essere fatto inviando traffico maligno al server (pacchetti molto grandi o corrotti).

La tecnica può essere espansa a più attaccanti (\textit{botnet}) dando vita a tecniche più sofisticate (come il \textbf{DDoS}, \textit{Distributed Denial of Service}).

\subsubsection{Intercezione di pacchetti}
Una problematica delle reti a commutazione di pacchetto è l'\textbf{intercezione} dei pacchetti instradati, o in inglese \textit{packet sniffing}.
Questo è più facile su reti wireless piuttosto che cablate, ed espone pericoli ovvi per la sicurezza (password, dati sensibili, ecc...).

Oggi, come vedremo, si usano protocolli che implementano forme di crittografia per mitigare questo tipo di problematiche.

\subsubsection{Spoofing IP}
Il problema di identità fasulle si presenta in Internet attraverso pacchetti malformati, con indirizzi sorgente sbagliati. 
Queste tecniche, seppure meno pericolose, possono comunque essere usate per confondere o comunque complicare il traffico sulle reti.

\end{document}

\documentclass[a4paper,11pt]{article}
\usepackage[a4paper, margin=8em]{geometry}

% usa i pacchetti per la scrittura in italiano
\usepackage[french,italian]{babel}
\usepackage[T1]{fontenc}
\usepackage[utf8]{inputenc}
\frenchspacing 

% usa i pacchetti per la formattazione matematica
\usepackage{amsmath, amssymb, amsthm, amsfonts}

% usa altri pacchetti
\usepackage{gensymb}
\usepackage{hyperref}
\usepackage{standalone}

\usepackage{colortbl}

\usepackage{xstring}
\usepackage{karnaugh-map}

% imposta il titolo
\title{Appunti Reti Informatiche}
\author{Luca Seggiani}
\date{2025}

% imposta lo stile
% usa helvetica
\usepackage[scaled]{helvet}
% usa palatino
\usepackage{palatino}
% usa un font monospazio guardabile
\usepackage{lmodern}

\renewcommand{\rmdefault}{ppl}
\renewcommand{\sfdefault}{phv}
\renewcommand{\ttdefault}{lmtt}

% circuiti
\usepackage{circuitikz}
\usetikzlibrary{babel}

% testo cerchiato
\newcommand*\circled[1]{\tikz[baseline=(char.base)]{
            \node[shape=circle,draw,inner sep=2pt] (char) {#1};}}

% disponi il titolo
\makeatletter
\renewcommand{\maketitle} {
	\begin{center} 
		\begin{minipage}[t]{.8\textwidth}
			\textsf{\huge\bfseries \@title} 
		\end{minipage}%
		\begin{minipage}[t]{.2\textwidth}
			\raggedleft \vspace{-1.65em}
			\textsf{\small \@author} \vfill
			\textsf{\small \@date}
		\end{minipage}
		\par
	\end{center}

	\thispagestyle{empty}
	\pagestyle{fancy}
}
\makeatother

% disponi teoremi
\usepackage{tcolorbox}
\newtcolorbox[auto counter, number within=section]{theorem}[2][]{%
	colback=blue!10, 
	colframe=blue!40!black, 
	sharp corners=northwest,
	fonttitle=\sffamily\bfseries, 
	title=Teorema~\thetcbcounter: #2, 
	#1
}

% disponi definizioni
\newtcolorbox[auto counter, number within=section]{definition}[2][]{%
	colback=red!10,
	colframe=red!40!black,
	sharp corners=northwest,
	fonttitle=\sffamily\bfseries,
	title=Definizione~\thetcbcounter: #2,
	#1
}

% disponi codice
\usepackage{listings}
\usepackage[table]{xcolor}

\definecolor{codegreen}{rgb}{0,0.6,0}
\definecolor{codegray}{rgb}{0.5,0.5,0.5}
\definecolor{codepurple}{rgb}{0.58,0,0.82}
\definecolor{backcolour}{rgb}{0.95,0.95,0.92}

\lstdefinestyle{codestyle}{
		backgroundcolor=\color{black!5}, 
		commentstyle=\color{codegreen},
		keywordstyle=\bfseries\color{magenta},
		numberstyle=\sffamily\tiny\color{black!60},
		stringstyle=\color{green!50!black},
		basicstyle=\ttfamily\footnotesize,
		breakatwhitespace=false,         
		breaklines=true,                 
		captionpos=b,                    
		keepspaces=true,                 
		numbers=left,                    
		numbersep=5pt,                  
		showspaces=false,                
		showstringspaces=false,
		showtabs=false,                  
		tabsize=2
}

\lstdefinestyle{shellstyle}{
		backgroundcolor=\color{black!5}, 
		basicstyle=\ttfamily\footnotesize\color{black}, 
		commentstyle=\color{black}, 
		keywordstyle=\color{black},
		numberstyle=\color{black!5},
		stringstyle=\color{black}, 
		showspaces=false,
		showstringspaces=false, 
		showtabs=false, 
		tabsize=2, 
		numbers=none, 
		breaklines=true
}


\lstdefinelanguage{assembler}{ 
  keywords={AAA, AAD, AAM, AAS, ADC, ADCB, ADCW, ADCL, ADD, ADDB, ADDW, ADDL, AND, ANDB, ANDW, ANDL,
        ARPL, BOUND, BSF, BSFL, BSFW, BSR, BSRL, BSRW, BSWAP, BT, BTC, BTCB, BTCW, BTCL, BTR, 
        BTRB, BTRW, BTRL, BTS, BTSB, BTSW, BTSL, CALL, CBW, CDQ, CLC, CLD, CLI, CLTS, CMC, CMP,
        CMPB, CMPW, CMPL, CMPS, CMPSB, CMPSD, CMPSW, CMPXCHG, CMPXCHGB, CMPXCHGW, CMPXCHGL,
        CMPXCHG8B, CPUID, CWDE, DAA, DAS, DEC, DECB, DECW, DECL, DIV, DIVB, DIVW, DIVL, ENTER,
        HLT, IDIV, IDIVB, IDIVW, IDIVL, IMUL, IMULB, IMULW, IMULL, IN, INB, INW, INL, INC, INCB,
        INCW, INCL, INS, INSB, INSD, INSW, INT, INT3, INTO, INVD, INVLPG, IRET, IRETD, JA, JAE,
        JB, JBE, JC, JCXZ, JE, JECXZ, JG, JGE, JL, JLE, JMP, JNA, JNAE, JNB, JNBE, JNC, JNE, JNG,
        JNGE, JNL, JNLE, JNO, JNP, JNS, JNZ, JO, JP, JPE, JPO, JS, JZ, LAHF, LAR, LCALL, LDS,
        LEA, LEAVE, LES, LFS, LGDT, LGS, LIDT, LMSW, LOCK, LODSB, LODSD, LODSW, LOOP, LOOPE,
        LOOPNE, LSL, LSS, LTR, MOV, MOVB, MOVW, MOVL, MOVSB, MOVSD, MOVSW, MOVSX, MOVSXB,
        MOVSXW, MOVSXL, MOVZX, MOVZXB, MOVZXW, MOVZXL, MUL, MULB, MULW, MULL, NEG, NEGB, NEGW,
        NEGL, NOP, NOT, NOTB, NOTW, NOTL, OR, ORB, ORW, ORL, OUT, OUTB, OUTW, OUTL, OUTSB, OUTSD,
        OUTSW, POP, POPL, POPW, POPB, POPA, POPAD, POPF, POPFD, PUSH, PUSHL, PUSHW, PUSHB, PUSHA, 
				PUSHAD, PUSHF, PUSHFD, RCL, RCLB, RCLW, MOVSL, MOVSB, MOVSW, STOSL, STOSB, STOSW, LODSB, LODSW,
				LODSL, INSB, INSW, INSL, OUTSB, OUTSL, OUTSW
        RCLL, RCR, RCRB, RCRW, RCRL, RDMSR, RDPMC, RDTSC, REP, REPE, REPNE, RET, ROL, ROLB, ROLW,
        ROLL, ROR, RORB, RORW, RORL, SAHF, SAL, SALB, SALW, SALL, SAR, SARB, SARW, SARL, SBB,
        SBBB, SBBW, SBBL, SCASB, SCASD, SCASW, SETA, SETAE, SETB, SETBE, SETC, SETE, SETG, SETGE,
        SETL, SETLE, SETNA, SETNAE, SETNB, SETNBE, SETNC, SETNE, SETNG, SETNGE, SETNL, SETNLE,
        SETNO, SETNP, SETNS, SETNZ, SETO, SETP, SETPE, SETPO, SETS, SETZ, SGDT, SHL, SHLB, SHLW,
        SHLL, SHLD, SHR, SHRB, SHRW, SHRL, SHRD, SIDT, SLDT, SMSW, STC, STD, STI, STOSB, STOSD,
        STOSW, STR, SUB, SUBB, SUBW, SUBL, TEST, TESTB, TESTW, TESTL, VERR, VERW, WAIT, WBINVD,
        XADD, XADDB, XADDW, XADDL, XCHG, XCHGB, XCHGW, XCHGL, XLAT, XLATB, XOR, XORB, XORW, XORL},
  keywordstyle=\color{blue}\bfseries,
  ndkeywordstyle=\color{darkgray}\bfseries,
  identifierstyle=\color{black},
  sensitive=false,
  comment=[l]{\#},
  morecomment=[s]{/*}{*/},
  commentstyle=\color{purple}\ttfamily,
  stringstyle=\color{red}\ttfamily,
  morestring=[b]',
  morestring=[b]"
}

\lstset{language=assembler, style=codestyle}

% disponi sezioni
\usepackage{titlesec}

\titleformat{\section}
	{\sffamily\Large\bfseries} 
	{\thesection}{1em}{} 
\titleformat{\subsection}
	{\sffamily\large\bfseries}   
	{\thesubsection}{1em}{} 
\titleformat{\subsubsection}
	{\sffamily\normalsize\bfseries} 
	{\thesubsubsection}{1em}{}

% tikz
\usepackage{tikz}

% float
\usepackage{float}

% grafici
\usepackage{pgfplots}
\pgfplotsset{width=10cm,compat=1.9}

% disponi alberi
\usepackage{forest}

\forestset{
	rectstyle/.style={
		for tree={rectangle,draw,font=\large\sffamily}
	},
	roundstyle/.style={
		for tree={circle,draw,font=\large}
	}
}

% disponi algoritmi
\usepackage{algorithm}
\usepackage{algorithmic}
\makeatletter
\renewcommand{\ALG@name}{Algoritmo}
\makeatother

% disponi numeri di pagina
\usepackage{fancyhdr}
\fancyhf{} 
\fancyfoot[L]{\sffamily{\thepage}}

\makeatletter
\fancyhead[L]{\raisebox{1ex}[0pt][0pt]{\sffamily{\@title \ \@date}}} 
\fancyhead[R]{\raisebox{1ex}[0pt][0pt]{\sffamily{\@author}}}
\makeatother

\begin{document}
% sezione (data)
\section{Lezione del 26-09-25}

% stili pagina
\thispagestyle{empty}
\pagestyle{fancy}

% testo
\subsection{Livelli di protocolli}
Analizzando l'architettura di Internet abbiamo introdotto il concetto di \textbf{protocollo}. 
Studiando il \textit{core} della rete abbiamo poi visto che la rete è molto complicata e costruita su più \textit{astrazioni}: host inviano pacchetti di applicazioni attraverso router su vari link, ecc...

Possiamo decidere di organizzare i protocolli che governano tali operazioni attraverso una struttura a \textit{layer} o \textbf{livelli}.
Ogni livello implementa un dato \textbf{servizio}, utilizzando le sue azioni interne, e basandosi sui servizi offerti dal livello sottostante.

Questo procedimento non è fuori luogo in sistemi complessi composti da diversi componenti in relazione fra di loro: permette infatti di gestire ogni componente singolarmente, senza doverci preoccupare di rompere la compatibilità con gli altri componenti (basta basarsi sullo stesso servizio sottostante e offrire lo stesso servizio al livello superiore).

Lo stack di protocolli internet che studiamo è quindi il seguente, a 5 livelli:
\begin{enumerate}
	\item \textbf{Application}: il livello che supporta le applicazioni in rete. Protocolli \textit{IMAP}, \textit{SMTP}, \textit{HTTP}; 
	\item \textbf{Transport}: il livello che supporta il trasferimento dati da processo a processo assicurando sicurezza e astrazione sul modello a pacchetto. Protocolli \textit{TCP} e \textit{UDP};
	\item \textbf{Network}: il livello che implementa il trasferimento di \textit{datagrammi} da sorgenti a destinazioni. Protocolli \textit{IP}, \textit{routing};
	\item \textbf{Link}: il livello che implementa la trasmissione di dati tra elementi di rete fra di loro "vicini". Protocolli \textit{Ethernet}, \textit{802.11 (WiFi)}, \textit{PPP};
	\item \textbf{Physical}: il livello fisico rappresentato dai bit sul mezzo di comunicazione.
\end{enumerate}

Un diverso stack è quello presentato dal modello di riferimento \textbf{ISO/OSI}. Questo è diverso dal modello presentato prima in quanto presenta qualche livello in più:
\begin{enumerate}
	\item \textbf{Application}: come sopra;
	\item \textbf{Presentation}: il livello che permette alle applicazioni di interpetare il significato dei dati. Protocolli di \textit{compressione}, \textit{crittografia}, ecc...;
	\item \textbf{Session}: il livello che permette \textit{sincronizzazione}, \textit{checkpoint}, \textit{recupero} di dati, ecc...;
	\item \textbf{Transport}: come sopra;
	\item \textbf{Network}: come sopra;
	\item \textbf{Link}: come sopra;
	\item \textbf{Physical}: come sopra;
\end{enumerate}

Queste funzioni, se strettamente necessarie, dovranno essere implementate nel livello application.

\subsubsection{Incapsulamento di protocolli}
Adesso che abbiamo stabilito una gerarchia di protocolli che permettono il collegamento internet, possiamo definire in maniera più precisa cosa accade quando ci colleghiamo, ad esempio, ad un werver HTTP per richiedere una pagina Web.

Dal nostro calcolatore, e in particolare dall'applicazione \textit{browser} in esecuzione sul nostro calcolatore, attraversiamo tutti i livelli (application, transport, network, link, e physical) per arrivare ad un router (probabilmente quello della rete di accesso). Man di mano che scendiamo in livelli più bassi corrediamo il messaggio inviato dall'applicazione con altre informazioni di controllo, utili ai livelli più bassi.
A questo punto il router ottiene il messaggio attraverso, magari solo il livello link e physical: questo procedimento si ripete, finché non raggiungiamo il server.
Quando il messaggio raggiunge la macchina server, risaliamo tutti i livelli (physical, link, network, transport, application) per arrivare all'applicazione server vera e propria, perdendo nel frattempo le informazioni aggiunte dai livelli sottostanti in fase di trasmissione.
A questo punto il server può interpretare il messaggio ed eventualmente rispondere.

\par\medskip

Con questa sezione abbiamo concluso l'introduzione al funzionamento (ad alto livello) di Internet, visto sia dall'\textit{edge} della rete (cioè dal punto di vista degli \textit{host}) che dal \textit{core} della rete (cioè dal punto di vista dell'\textit{infrastruttura}) di rete.

\subsection{Applicazioni in rete}
Veniamo quindi allo sviluppo di \textbf{applicazioni} in rete, con l'obiettivo di tornare alle specifiche delle reti in un secondo momento.

In questa sezione vedremo i principi delle applicazioni Web \textit{client-server} e \textit{peer-to-peer} (\textbf{P2P}), in particolare approfondendo i protocolli Web (HTTP), e-mail (SMTP, IMAP), il sistema DNS, nonché la programmazione con l'API dei \textit{socket} coi protocolli TCP e UDP.

Nostro focus sarà quindi il livello applicazione (e in minor parte di trasporto), e gli aspetti concettuali e di implementazione di applicazioni in rete.

Creare un'applicazione in rete significa scrivere programmi che:
\begin{itemize}
	\item Girano su diversi sistemi;
	\item Comunicano via la rete.
\end{itemize}

Le applicazioni vengono scritte per gli \textbf{host}: i router non eseguono applicazioni utente, e anzi \textit{non eseguono} nemmeno lo stack protocollare completo (si limitano al livello link e al massimo network).
Sviluppare applicazioni per i sistemi all'\textit{edge} della rete permette invece la facile e rapida propagazione delle stesse.

\subsubsection{Paradigma client-server}
Il paradigma \textbf{client-server} è il più comune per le applicazioni in rete.
In questo caso individuiamo due agenti principali:
\begin{itemize}
	\item Il \textbf{server} è un host sempre attivo, con indirizzo IP permanente, solitamente distribuito su data center, per permettere scalabilità (qui si sfruttano tecnologie come \textit{load balancer}, ecc...);
	\item Il \textbf{client} è un host che stabilisce contatto intermittente col sever, che può avere indirizzo IP variabile, e che non interagisce mai direttamente con altri client.
\end{itemize}

Abbiamo quindi che l'identità di client e server è \textit{forte}, la relazione fra i due è asimmetrica e ben definita.
In termini di risorse, il client non dovrà avere particolari risorse computazionali, mentre il server dovrà essere capace di gestire le richieste di tutti i client.

Esempi di protocolli client-server sono il protocollo HTTP, i protocolli di posta IMAP e il protocollo di trasferimento file FTP. 

\subsubsection{Paradigma peer-to-peer}
Nel paradigma \textbf{peer-to-peer} non esiste un singolo server always-on, ma ogni peer può comportarsi in modalità intermittente sia da client che da server.
Questo significa che i peer ricevono servizi da altri peer, fornendo in cambio altri servizi.

Questa caratteristica permettte l'\textit{auto-scalabilità}: la rete P2P si sviluppa autonomamente man di mano che si aggiungono peer.

I peer hanno un identità molto più labile rispetto a quelle di client e server tradizionali: l'indirizzo IP può essere dinamico, non sono sempre online, il loro servizio potrebbe essere intermittente, ecc...

\subsubsection{Socket}
I \textbf{socket} sono l'API che implementa la connettività di rete per le applicazioni che scriveremo. Sono offerti dal sistema operativo e rappresentano un'astrazione per la connessione di rete (come i file rappresentano un'astrazione per il disco).

L'analogia tipica del socket è quella di una \textit{porta}.
I messaggi entrano dalla porta ed escono dalla porta: quello che sta dietro alla porta è parte dell'infrastruttura implementata prima dal sistema operativo (che implementa i livelli protocollari) e poi dalla rete Internet in sé per sé, ed è astratto via dal programmatore.



\end{document}

\documentclass[a4paper,11pt]{article}
\usepackage[a4paper, margin=8em]{geometry}

% usa i pacchetti per la scrittura in italiano
\usepackage[french,italian]{babel}
\usepackage[T1]{fontenc}
\usepackage[utf8]{inputenc}
\frenchspacing 

% usa i pacchetti per la formattazione matematica
\usepackage{amsmath, amssymb, amsthm, amsfonts}

% usa altri pacchetti
\usepackage{gensymb}
\usepackage{hyperref}
\usepackage{standalone}

\usepackage{colortbl}

\usepackage{xstring}
\usepackage{karnaugh-map}

% imposta il titolo
\title{Appunti Reti Informatiche}
\author{Luca Seggiani}
\date{2025}

% imposta lo stile
% usa helvetica
\usepackage[scaled]{helvet}
% usa palatino
\usepackage{palatino}
% usa un font monospazio guardabile
\usepackage{lmodern}

\renewcommand{\rmdefault}{ppl}
\renewcommand{\sfdefault}{phv}
\renewcommand{\ttdefault}{lmtt}

% circuiti
\usepackage{circuitikz}
\usetikzlibrary{babel}

% testo cerchiato
\newcommand*\circled[1]{\tikz[baseline=(char.base)]{
            \node[shape=circle,draw,inner sep=2pt] (char) {#1};}}

% disponi il titolo
\makeatletter
\renewcommand{\maketitle} {
	\begin{center} 
		\begin{minipage}[t]{.8\textwidth}
			\textsf{\huge\bfseries \@title} 
		\end{minipage}%
		\begin{minipage}[t]{.2\textwidth}
			\raggedleft \vspace{-1.65em}
			\textsf{\small \@author} \vfill
			\textsf{\small \@date}
		\end{minipage}
		\par
	\end{center}

	\thispagestyle{empty}
	\pagestyle{fancy}
}
\makeatother

% disponi teoremi
\usepackage{tcolorbox}
\newtcolorbox[auto counter, number within=section]{theorem}[2][]{%
	colback=blue!10, 
	colframe=blue!40!black, 
	sharp corners=northwest,
	fonttitle=\sffamily\bfseries, 
	title=Teorema~\thetcbcounter: #2, 
	#1
}

% disponi definizioni
\newtcolorbox[auto counter, number within=section]{definition}[2][]{%
	colback=red!10,
	colframe=red!40!black,
	sharp corners=northwest,
	fonttitle=\sffamily\bfseries,
	title=Definizione~\thetcbcounter: #2,
	#1
}

% disponi codice
\usepackage{listings}
\usepackage[table]{xcolor}

\definecolor{codegreen}{rgb}{0,0.6,0}
\definecolor{codegray}{rgb}{0.5,0.5,0.5}
\definecolor{codepurple}{rgb}{0.58,0,0.82}
\definecolor{backcolour}{rgb}{0.95,0.95,0.92}

\lstdefinestyle{codestyle}{
		backgroundcolor=\color{black!5}, 
		commentstyle=\color{codegreen},
		keywordstyle=\bfseries\color{magenta},
		numberstyle=\sffamily\tiny\color{black!60},
		stringstyle=\color{green!50!black},
		basicstyle=\ttfamily\footnotesize,
		breakatwhitespace=false,         
		breaklines=true,                 
		captionpos=b,                    
		keepspaces=true,                 
		numbers=left,                    
		numbersep=5pt,                  
		showspaces=false,                
		showstringspaces=false,
		showtabs=false,                  
		tabsize=2
}

\lstdefinestyle{shellstyle}{
		backgroundcolor=\color{black!5}, 
		basicstyle=\ttfamily\footnotesize\color{black}, 
		commentstyle=\color{black}, 
		keywordstyle=\color{black},
		numberstyle=\color{black!5},
		stringstyle=\color{black}, 
		showspaces=false,
		showstringspaces=false, 
		showtabs=false, 
		tabsize=2, 
		numbers=none, 
		breaklines=true
}


\lstdefinelanguage{assembler}{ 
  keywords={AAA, AAD, AAM, AAS, ADC, ADCB, ADCW, ADCL, ADD, ADDB, ADDW, ADDL, AND, ANDB, ANDW, ANDL,
        ARPL, BOUND, BSF, BSFL, BSFW, BSR, BSRL, BSRW, BSWAP, BT, BTC, BTCB, BTCW, BTCL, BTR, 
        BTRB, BTRW, BTRL, BTS, BTSB, BTSW, BTSL, CALL, CBW, CDQ, CLC, CLD, CLI, CLTS, CMC, CMP,
        CMPB, CMPW, CMPL, CMPS, CMPSB, CMPSD, CMPSW, CMPXCHG, CMPXCHGB, CMPXCHGW, CMPXCHGL,
        CMPXCHG8B, CPUID, CWDE, DAA, DAS, DEC, DECB, DECW, DECL, DIV, DIVB, DIVW, DIVL, ENTER,
        HLT, IDIV, IDIVB, IDIVW, IDIVL, IMUL, IMULB, IMULW, IMULL, IN, INB, INW, INL, INC, INCB,
        INCW, INCL, INS, INSB, INSD, INSW, INT, INT3, INTO, INVD, INVLPG, IRET, IRETD, JA, JAE,
        JB, JBE, JC, JCXZ, JE, JECXZ, JG, JGE, JL, JLE, JMP, JNA, JNAE, JNB, JNBE, JNC, JNE, JNG,
        JNGE, JNL, JNLE, JNO, JNP, JNS, JNZ, JO, JP, JPE, JPO, JS, JZ, LAHF, LAR, LCALL, LDS,
        LEA, LEAVE, LES, LFS, LGDT, LGS, LIDT, LMSW, LOCK, LODSB, LODSD, LODSW, LOOP, LOOPE,
        LOOPNE, LSL, LSS, LTR, MOV, MOVB, MOVW, MOVL, MOVSB, MOVSD, MOVSW, MOVSX, MOVSXB,
        MOVSXW, MOVSXL, MOVZX, MOVZXB, MOVZXW, MOVZXL, MUL, MULB, MULW, MULL, NEG, NEGB, NEGW,
        NEGL, NOP, NOT, NOTB, NOTW, NOTL, OR, ORB, ORW, ORL, OUT, OUTB, OUTW, OUTL, OUTSB, OUTSD,
        OUTSW, POP, POPL, POPW, POPB, POPA, POPAD, POPF, POPFD, PUSH, PUSHL, PUSHW, PUSHB, PUSHA, 
				PUSHAD, PUSHF, PUSHFD, RCL, RCLB, RCLW, MOVSL, MOVSB, MOVSW, STOSL, STOSB, STOSW, LODSB, LODSW,
				LODSL, INSB, INSW, INSL, OUTSB, OUTSL, OUTSW
        RCLL, RCR, RCRB, RCRW, RCRL, RDMSR, RDPMC, RDTSC, REP, REPE, REPNE, RET, ROL, ROLB, ROLW,
        ROLL, ROR, RORB, RORW, RORL, SAHF, SAL, SALB, SALW, SALL, SAR, SARB, SARW, SARL, SBB,
        SBBB, SBBW, SBBL, SCASB, SCASD, SCASW, SETA, SETAE, SETB, SETBE, SETC, SETE, SETG, SETGE,
        SETL, SETLE, SETNA, SETNAE, SETNB, SETNBE, SETNC, SETNE, SETNG, SETNGE, SETNL, SETNLE,
        SETNO, SETNP, SETNS, SETNZ, SETO, SETP, SETPE, SETPO, SETS, SETZ, SGDT, SHL, SHLB, SHLW,
        SHLL, SHLD, SHR, SHRB, SHRW, SHRL, SHRD, SIDT, SLDT, SMSW, STC, STD, STI, STOSB, STOSD,
        STOSW, STR, SUB, SUBB, SUBW, SUBL, TEST, TESTB, TESTW, TESTL, VERR, VERW, WAIT, WBINVD,
        XADD, XADDB, XADDW, XADDL, XCHG, XCHGB, XCHGW, XCHGL, XLAT, XLATB, XOR, XORB, XORW, XORL},
  keywordstyle=\color{blue}\bfseries,
  ndkeywordstyle=\color{darkgray}\bfseries,
  identifierstyle=\color{black},
  sensitive=false,
  comment=[l]{\#},
  morecomment=[s]{/*}{*/},
  commentstyle=\color{purple}\ttfamily,
  stringstyle=\color{red}\ttfamily,
  morestring=[b]',
  morestring=[b]"
}

\lstset{language=assembler, style=codestyle}

% disponi sezioni
\usepackage{titlesec}

\titleformat{\section}
	{\sffamily\Large\bfseries} 
	{\thesection}{1em}{} 
\titleformat{\subsection}
	{\sffamily\large\bfseries}   
	{\thesubsection}{1em}{} 
\titleformat{\subsubsection}
	{\sffamily\normalsize\bfseries} 
	{\thesubsubsection}{1em}{}

% tikz
\usepackage{tikz}

% float
\usepackage{float}

% grafici
\usepackage{pgfplots}
\pgfplotsset{width=10cm,compat=1.9}

% disponi alberi
\usepackage{forest}

\forestset{
	rectstyle/.style={
		for tree={rectangle,draw,font=\large\sffamily}
	},
	roundstyle/.style={
		for tree={circle,draw,font=\large}
	}
}

% disponi algoritmi
\usepackage{algorithm}
\usepackage{algorithmic}
\makeatletter
\renewcommand{\ALG@name}{Algoritmo}
\makeatother

% disponi numeri di pagina
\usepackage{fancyhdr}
\fancyhf{} 
\fancyfoot[L]{\sffamily{\thepage}}

\makeatletter
\fancyhead[L]{\raisebox{1ex}[0pt][0pt]{\sffamily{\@title \ \@date}}} 
\fancyhead[R]{\raisebox{1ex}[0pt][0pt]{\sffamily{\@author}}}
\makeatother

\begin{document}
% sezione (data)
\section{Lezione del 29-09-25}

% stili pagina
\thispagestyle{empty}
\pagestyle{fancy}

% testo
Continuiamo a vedere nel dettaglio l'ambito delo \textit{sviluppo di applicazioni}.

\subsection{Comunicazione fra processi}
Abbiamo visto come, sebbene debbano girare su una vasta gamma di dispositivi ed interagire con svariate tecnologie di collegamento, queste possono essere riassunte nei paradigmi \text{client-server} e \textit{peer-to-peer}.
Inoltre, abbiamo visto come maggior parte della complessità dell'infrastruttura di rete sia astratta via nei moderni S/O dietro il meccanismo dei \textit{socket}.

In particolare, vediamo i nostri applicativi come composti da \textbf{processi} in esecuzione su macchine fisiche, da cui \textbf{processi client} su macchine client e \textbf{processi server} su macchine server.
Le modalità secondo la quale questi processi si scambieranno \textbf{messaggi} saranno rappresentate dalla cosiddetta \textbf{IPC} (\textit{Inter Process Communication}).

Notiamo che non è necessario che la IPC sia su macchine diverse, due processi nella stessa macchina possono infatti comunicare fra di loro come se si trovassero su macchine diverse collegate in rete.

Quello che vanno ad implementare i \textbf{socket}, standard \textit{de facto} apparso in origine nei sistemi operativi BSD, è appunto l'IPC fra più processi.

\subsubsection{Modalità di indirizzamento}
Per ricevere messaggi, i processi devono avere degli \textit{identificatori}.
Questi sono rappresentati da \textbf{indirizzi IP} su 32 bit.
L'indirizzo IP della macchina su cui i processi girano non basta, in quanto chiaramente una macchina può avere più di un processo in esecuzione.

\subsubsection{Protocolli livello application}
Un protocollo di livello \textit{application} dovrà a questo punto definire diverse specifiche sul'IPC:
\begin{itemize}
	\item Il \textbf{tipo} di messaggi scambiati: questi possono solitamente essere \textit{richieste} e \textit{risposte};
	\item \textbf{Sintassi} dei messaggi: quali campi presentano i messaggi e come i campi sono delineati nella struttura del messaggio;
	\item \textbf{Semantica} dei messaggi: il significato dell'informazione contenuta nei campi;
	\item \textbf{Regole}: su come e quando i processi possono ricevere o inviare messaggi.
\end{itemize}

Esistono diversi protocolli application, sia \textbf{aperti} (interoperabili per applicazioni di più sviluppatori, lo sono il protcollo HTTP, i protocolli di posta, ecc...) che \textbf{proprietari} (lo sono ad esempio i protocolli associati a prodotti software proprietari come Skype, ecc...).
In particolare, i protocolli aperti possono essere sia \textbf{ufficiali} (supportati da agenzie come l'RFC), o \textbf{non ufficiali} (basati su documenti non ufficiali ma di pubblico accesso, diventati standard \textit{de facto} dopo grande adozione, ad esempio BitTorrent).

\subsubsection{Servizi ad applicazioni}
Iniziamo a vedere quali tipi di servizi potrebbero servire ad un applicazione in rete.

\begin{itemize}
	\item \textbf{Integrità dati}: alcune applicazioni (trasferimento dati, transazioni, ecc...) potrebbero richiedere un trasferimento sicuro al 100\%. Altre (videogiochi, streaming, ecc...) potrebbero invece poter tollerare perdite paraziale dei dati in fase di trasmissione.
	\item \textbf{Throughput}: alcune applicazioni (multimedia, ecc...) richiedono una quantità minima di throughput per essere efficaci. Altre (le cosiddette applicazioni \textit{elastiche}) ne prendono quanto non hanno a disposizione.
	\item \textbf{Tempo}: alcune applicazioni (ancora videogiochi, telefonia online, ecc...) potrebbero richiedere delay temporali molto contenuti per essere efficaci.
	\item \textbf{Sicurezza}: servizi più o meno critici richiedono livelli variabili di sicurezza (crittografia, ecc...).
\end{itemize}

Su questa base, potremmo classificare alcune applicazioni sulla base dei requisiti di trasporto che hanno:
\begin{table}[H]
	\center \rowcolors{2}{white}{black!10}
	\begin{tabular} { p{3cm} || p{2.5cm} | p{2.5cm} | p{2.5cm} | p{2.5cm} }
		\bfseries Applicazione & \bfseries Perdita dati & \bfseries Throughput & \bfseries Tempo & \bfseries Sicurezza \\ 
		\hline
		Trasferimento file/download & Nessuna & Elastico & Indifferente & Dipende \\
		E-mail & Nessuna & Elastico & Indifferente & Dipende \\ 
		Web & Nessuna & Elastico & Indifferente & Dipende \\ 
		Streaming & Tollerante & 5 Kbps - 5 Mbps & Pochi secondi & Non importante \\ 
		Videogiochi & Tollerante & 5 Kbps - 5 Mbps & Pochi millisecondi & Dipende \\ 
		Messaggistica istantanea & Nessuna & Elastico & Perlopiù indifferente & Critica \\ 
	\end{tabular}
\end{table}

\subsubsection{Protocolli livello transport}
Potrebbe essere utile, per capire come realizzare le specifiche sopra descritte, studiare ad alto livello i due protocolli di trasporto principali:
\begin{itemize}
	\item \textbf{TCP} (\textit{Transmission Control Protocol}): assicura il trasporto \textit{affidabile} di messaggi tra processi, controllo del \textit{flusso} e delle \textit{congestioni}, ma ha promesse più scarse nel campo della temporizzazione e del throughput (sebbene assicuri un \textit{throughput minimo}).
		Orientato alla \textit{connessione} fra processi client e server, è più sicuro dell'alternativa;
	\item \textbf{UDP} (\textit{User Datagram Protocol}): meno sicuro e affidabile, non fornisce servizi di controllo flussi o congestioni, né throughput minimo. Questo lo rende adatto per soluzioni a basso overhead, dove le prestazioni sono più significative della correttezza dei dati.
\end{itemize}

Il controllo del flusso e delle congestioni si collega a quanto visto nella sezione 2.2:
\begin{itemize}
	\item Il controllo del \textbf{flusso} si assicura che il mittente non potrà sopraffare il destinatario con una mole troppo grande di messaggi;
	\item Il controllo delle \textbf{congestioni} assicura che i ruoter nell'infrastruttura fra mittente e destinatario non vengano sovraccaricati, o se lo sono l'informazione venga deviata in modo da assicurare che i pacchetti arrivino in maniera affidabile. 
\end{itemize}

\par\smallskip

Possiamo quindi assegnare ad ognuna delle applicazioni viste nella tabella di sezione 4.13 un protocollo adatto:
\begin{table}[H]
	\center \rowcolors{2}{white}{black!10}
	\begin{tabular} { p{3cm} || p{5cm} | p{5cm} }
		\bfseries Applicazione & \bfseries Protocollo application & \bfseries Protocollo transport \\ 
		\hline
		Trasferimento file/download & FTP & TCP \\
		E-mail & SMTP & TCP \\ 
		Web & HTTP & TCP \\ 
		Streaming & HTTP, DASH & TCP \\ 
		Videogiochi & WOW, FPS o proprietario & TCP o UDP \\ 
		Messaggistica istantanea & HTTP o proprietario & TCP o UDP (telefonia) \\ 
	\end{tabular}
\end{table}

\subsubsection{Sicurezza in TCP}
I socket TCP e UDP di base non forniscono particolari funzionalità di sicurezza: i dati sono trasmessi senza crittografia, per cui dati sensibili sono visibili in chiaro.

Si può sfruttare il protocollo (in verità protocollo \textit{middleware}, che sta fra livello transport e livello application) \textbf{TLS} (\textit{Transport Layer Security}) per fornire connessioni TCP crittografate, con integrità dei dati assicurata e autenticazione del destinatario.

Possiamo sfruttare TLS in 2 modi principali:
\begin{itemize}
	\item TSL implementato in applicazione, che a sua volta interagisce con socket TCP;
	\item API che fornisce socket TLS, che ricevono dati in chiaro e li inviano su Internet crittografati.
\end{itemize}

La manifestazione più comune del protocollo TLS si vede negli URL delle pagine web, che iniziano con \lstinline|http://| quando si usa HTTP su TCP puro, e con \lstinline|https://| quando si usa HTTP su TCP con TLS.

\subsection{Web e HTTP}
Veniamo quindi a dettagliare la più famosa applicazione sviluppata su Internet, cioè il Web.
Come abbiamo visto il Web è supportato dal protocollo \textbf{HTTP} (\textit{HyperText Transfer Protocol}).

Ricordiamo quindi che una \textbf{pagina} Web consiste di oggetti di diversi formati che possono essere allocati su più Web server.
Le pagine in sé per se sono anch'esse oggetti, consistono di file \textbf{HTML} (\textit{Hypertext Markup Language}), indirizzabili assieme come tutti gli altri oggetti che le compongono da un \textbf{URL} (\textit{Uniform Resource Locator}), in forma:
\begin{lstlisting}[language=html, style=codestyle]	
<schema>//<nome-host>/<percorso_risorsa>
\end{lstlisting}
ad esempio, \lstinline|https://www.bittorrent.org/index.html|.

\subsubsection{Protocollo HTTP}
Il protocollo HTTP è basato sul modello client-server (richiede un protocollo di livello transport che supporti connessioni client-server, quasi sempre TCP).
In questo, il client è rappresentato dal \textit{browser}, che compila richieste per ottenere pagine web, mentre il server è rappresentato dal \textit{Web server}, che riceve le richieste dei browser e risponde inviando oggetti.

Nello specifico, una connessione HTTP si svolge come segue:
\begin{itemize}
	\item Il client inizia la connessione TCP (lato applicazione, crea il socket) col server, alla porta 80;
	\item Il server accetta la connessione TCP del client;
	\item Messaggi HTTP vengono scambiati fra client (browser) e server sulla linea TCP;
	\item La connessione TCP viene chiusa.
\end{itemize}

Il protocollo HTTP è \textit{privo di stato}, cioè non si mantiene nessuna informazione riguardo alle richieste passate del client.

Esistono 2 tipi di connessioni HTTP:
\begin{itemize}
	\item HTTP \textbf{non persistente}: si apre la connessione TCP e si invia al più un oggetto sulla connessione prima di chiudere. In questo caso scaricare più oggetti richiede più connessioni TCP; 
	\item HTTP \textbf{persistente}: si apre la connessione TCP e si inviano più oggetti sulla connessione prima di chiudere.
\end{itemize}

Definiamo il \textbf{RTT} (\textit{Round-Trip Time}) come il tempo che un piccolo pacchetto impiega per viaggiare da client a server e ritorno.

\begin{itemize}
	\item 
Nel caso dell'HTTP non persistente, richiediamo un RTT per iniziare la connessione TCP, un RTT per richiedere il file, più il tempo necessario a trasferire il file vero e proprio, per cui si impiega:
$$
T_{\text{non-pers}} = 2 \text{RTT} + T_{\text{tras}}
$$
per ottenere ogni file.
	\item
Nel caso dell'HTTP persistente, dovremmo comunque usare 2 RTT per iniziare la connessione e chiedere il primo file, ma ogni file successivo richiederà solamente il tempo RTT necessario a richiedere la risorsa, per cui risparmieremo tempo.

Chiaramente in questo caso avremo il problema di dover capire \textit{quando} chiudere la trasmissione: magari dopo $n$ richieste, dopo un tempo $T$ (\textit{Timeout}), ecc...
\end{itemize}

\subsubsection{Richieste HTTP}
Abbiamo visto come i messaggi HTTP appartengono a 2 tipi, \textit{richieste} e \textit{risposte}.
I messaggi sono in formato ASCII, quindi leggibile dall'uomo.

In particolare, l'header di una richiesta HTTP ha la seguente forma generale:
\begin{lstlisting}[language=html,style=codestyle]	
GET /index.html HTTP/1.1
Host: www-net.cs.umass.edu
User-Agent: Firefox/3.6.10
Accept: text/html,application/xhtml+xml
Accept-Language: en-US,en
Accept-Encoding: gzip, deflate
Connection: keep-alive
\end{lstlisting}

Ogni riga è terminata da \lstinline|\r\n|, cartteri di ritorno carrello e nuova linea, non riportati nell'esempio.

La prima riga definisce il \textbf{tipo} di richiesta (GET, POST, HEAD, ecc...), la \textbf{risorsa} richiesta e la \textbf{versione} del protocollo che il client vuole utilizzare. 
La seconda linea contiene poi l'host del server richiesto, e la terza il processo (qui il browser Firefox) che effettua la richiesta.
Le successive rige definiscono il tipo di oggetto che il client è disposto ad ottenere (tipo MIME, lingua, codifica e compressione, ecc...).
Infine, l'ultima riga stabilisce le regole di connessione richieste dal client, in questo caso \lstinline|keep-alive| (quindi HTTP persistente).

Un doppio ritorno carrello e nuova linea segnala la fine delle linee di header e l'inizio dei dati veri e propri.

\end{document}

\documentclass[a4paper,11pt]{article}
\usepackage[a4paper, margin=8em]{geometry}

% usa i pacchetti per la scrittura in italiano
\usepackage[french,italian]{babel}
\usepackage[T1]{fontenc}
\usepackage[utf8]{inputenc}
\frenchspacing 

% usa i pacchetti per la formattazione matematica
\usepackage{amsmath, amssymb, amsthm, amsfonts}

% usa altri pacchetti
\usepackage{gensymb}
\usepackage{hyperref}
\usepackage{standalone}

\usepackage{colortbl}

\usepackage{xstring}
\usepackage{karnaugh-map}

% imposta il titolo
\title{Appunti Reti Informatiche}
\author{Luca Seggiani}
\date{2025}

% imposta lo stile
% usa helvetica
\usepackage[scaled]{helvet}
% usa palatino
\usepackage{palatino}
% usa un font monospazio guardabile
\usepackage{lmodern}

\renewcommand{\rmdefault}{ppl}
\renewcommand{\sfdefault}{phv}
\renewcommand{\ttdefault}{lmtt}

% circuiti
\usepackage{circuitikz}
\usetikzlibrary{babel}

% testo cerchiato
\newcommand*\circled[1]{\tikz[baseline=(char.base)]{
            \node[shape=circle,draw,inner sep=2pt] (char) {#1};}}

% disponi il titolo
\makeatletter
\renewcommand{\maketitle} {
	\begin{center} 
		\begin{minipage}[t]{.8\textwidth}
			\textsf{\huge\bfseries \@title} 
		\end{minipage}%
		\begin{minipage}[t]{.2\textwidth}
			\raggedleft \vspace{-1.65em}
			\textsf{\small \@author} \vfill
			\textsf{\small \@date}
		\end{minipage}
		\par
	\end{center}

	\thispagestyle{empty}
	\pagestyle{fancy}
}
\makeatother

% disponi teoremi
\usepackage{tcolorbox}
\newtcolorbox[auto counter, number within=section]{theorem}[2][]{%
	colback=blue!10, 
	colframe=blue!40!black, 
	sharp corners=northwest,
	fonttitle=\sffamily\bfseries, 
	title=Teorema~\thetcbcounter: #2, 
	#1
}

% disponi definizioni
\newtcolorbox[auto counter, number within=section]{definition}[2][]{%
	colback=red!10,
	colframe=red!40!black,
	sharp corners=northwest,
	fonttitle=\sffamily\bfseries,
	title=Definizione~\thetcbcounter: #2,
	#1
}

% disponi codice
\usepackage{listings}
\usepackage[table]{xcolor}

\definecolor{codegreen}{rgb}{0,0.6,0}
\definecolor{codegray}{rgb}{0.5,0.5,0.5}
\definecolor{codepurple}{rgb}{0.58,0,0.82}
\definecolor{backcolour}{rgb}{0.95,0.95,0.92}

\lstdefinestyle{codestyle}{
		backgroundcolor=\color{black!5}, 
		commentstyle=\color{codegreen},
		keywordstyle=\bfseries\color{magenta},
		numberstyle=\sffamily\tiny\color{black!60},
		stringstyle=\color{green!50!black},
		basicstyle=\ttfamily\footnotesize,
		breakatwhitespace=false,         
		breaklines=true,                 
		captionpos=b,                    
		keepspaces=true,                 
		numbers=left,                    
		numbersep=5pt,                  
		showspaces=false,                
		showstringspaces=false,
		showtabs=false,                  
		tabsize=2
}

\lstdefinestyle{shellstyle}{
		backgroundcolor=\color{black!5}, 
		basicstyle=\ttfamily\footnotesize\color{black}, 
		commentstyle=\color{black}, 
		keywordstyle=\color{black},
		numberstyle=\color{black!5},
		stringstyle=\color{black}, 
		showspaces=false,
		showstringspaces=false, 
		showtabs=false, 
		tabsize=2, 
		numbers=none, 
		breaklines=true
}


\lstdefinelanguage{assembler}{ 
  keywords={AAA, AAD, AAM, AAS, ADC, ADCB, ADCW, ADCL, ADD, ADDB, ADDW, ADDL, AND, ANDB, ANDW, ANDL,
        ARPL, BOUND, BSF, BSFL, BSFW, BSR, BSRL, BSRW, BSWAP, BT, BTC, BTCB, BTCW, BTCL, BTR, 
        BTRB, BTRW, BTRL, BTS, BTSB, BTSW, BTSL, CALL, CBW, CDQ, CLC, CLD, CLI, CLTS, CMC, CMP,
        CMPB, CMPW, CMPL, CMPS, CMPSB, CMPSD, CMPSW, CMPXCHG, CMPXCHGB, CMPXCHGW, CMPXCHGL,
        CMPXCHG8B, CPUID, CWDE, DAA, DAS, DEC, DECB, DECW, DECL, DIV, DIVB, DIVW, DIVL, ENTER,
        HLT, IDIV, IDIVB, IDIVW, IDIVL, IMUL, IMULB, IMULW, IMULL, IN, INB, INW, INL, INC, INCB,
        INCW, INCL, INS, INSB, INSD, INSW, INT, INT3, INTO, INVD, INVLPG, IRET, IRETD, JA, JAE,
        JB, JBE, JC, JCXZ, JE, JECXZ, JG, JGE, JL, JLE, JMP, JNA, JNAE, JNB, JNBE, JNC, JNE, JNG,
        JNGE, JNL, JNLE, JNO, JNP, JNS, JNZ, JO, JP, JPE, JPO, JS, JZ, LAHF, LAR, LCALL, LDS,
        LEA, LEAVE, LES, LFS, LGDT, LGS, LIDT, LMSW, LOCK, LODSB, LODSD, LODSW, LOOP, LOOPE,
        LOOPNE, LSL, LSS, LTR, MOV, MOVB, MOVW, MOVL, MOVSB, MOVSD, MOVSW, MOVSX, MOVSXB,
        MOVSXW, MOVSXL, MOVZX, MOVZXB, MOVZXW, MOVZXL, MUL, MULB, MULW, MULL, NEG, NEGB, NEGW,
        NEGL, NOP, NOT, NOTB, NOTW, NOTL, OR, ORB, ORW, ORL, OUT, OUTB, OUTW, OUTL, OUTSB, OUTSD,
        OUTSW, POP, POPL, POPW, POPB, POPA, POPAD, POPF, POPFD, PUSH, PUSHL, PUSHW, PUSHB, PUSHA, 
				PUSHAD, PUSHF, PUSHFD, RCL, RCLB, RCLW, MOVSL, MOVSB, MOVSW, STOSL, STOSB, STOSW, LODSB, LODSW,
				LODSL, INSB, INSW, INSL, OUTSB, OUTSL, OUTSW
        RCLL, RCR, RCRB, RCRW, RCRL, RDMSR, RDPMC, RDTSC, REP, REPE, REPNE, RET, ROL, ROLB, ROLW,
        ROLL, ROR, RORB, RORW, RORL, SAHF, SAL, SALB, SALW, SALL, SAR, SARB, SARW, SARL, SBB,
        SBBB, SBBW, SBBL, SCASB, SCASD, SCASW, SETA, SETAE, SETB, SETBE, SETC, SETE, SETG, SETGE,
        SETL, SETLE, SETNA, SETNAE, SETNB, SETNBE, SETNC, SETNE, SETNG, SETNGE, SETNL, SETNLE,
        SETNO, SETNP, SETNS, SETNZ, SETO, SETP, SETPE, SETPO, SETS, SETZ, SGDT, SHL, SHLB, SHLW,
        SHLL, SHLD, SHR, SHRB, SHRW, SHRL, SHRD, SIDT, SLDT, SMSW, STC, STD, STI, STOSB, STOSD,
        STOSW, STR, SUB, SUBB, SUBW, SUBL, TEST, TESTB, TESTW, TESTL, VERR, VERW, WAIT, WBINVD,
        XADD, XADDB, XADDW, XADDL, XCHG, XCHGB, XCHGW, XCHGL, XLAT, XLATB, XOR, XORB, XORW, XORL},
  keywordstyle=\color{blue}\bfseries,
  ndkeywordstyle=\color{darkgray}\bfseries,
  identifierstyle=\color{black},
  sensitive=false,
  comment=[l]{\#},
  morecomment=[s]{/*}{*/},
  commentstyle=\color{purple}\ttfamily,
  stringstyle=\color{red}\ttfamily,
  morestring=[b]',
  morestring=[b]"
}

\lstset{language=assembler, style=codestyle}

% disponi sezioni
\usepackage{titlesec}

\titleformat{\section}
	{\sffamily\Large\bfseries} 
	{\thesection}{1em}{} 
\titleformat{\subsection}
	{\sffamily\large\bfseries}   
	{\thesubsection}{1em}{} 
\titleformat{\subsubsection}
	{\sffamily\normalsize\bfseries} 
	{\thesubsubsection}{1em}{}

% tikz
\usepackage{tikz}

% float
\usepackage{float}

% grafici
\usepackage{pgfplots}
\pgfplotsset{width=10cm,compat=1.9}

% disponi alberi
\usepackage{forest}

\forestset{
	rectstyle/.style={
		for tree={rectangle,draw,font=\large\sffamily}
	},
	roundstyle/.style={
		for tree={circle,draw,font=\large}
	}
}

% disponi algoritmi
\usepackage{algorithm}
\usepackage{algorithmic}
\makeatletter
\renewcommand{\ALG@name}{Algoritmo}
\makeatother

% disponi numeri di pagina
\usepackage{fancyhdr}
\fancyhf{} 
\fancyfoot[L]{\sffamily{\thepage}}

\makeatletter
\fancyhead[L]{\raisebox{1ex}[0pt][0pt]{\sffamily{\@title \ \@date}}} 
\fancyhead[R]{\raisebox{1ex}[0pt][0pt]{\sffamily{\@author}}}
\makeatother

\begin{document}
% sezione (data)
\section{Lezione del 01-10-25}

% stili pagina
\thispagestyle{empty}
\pagestyle{fancy}

% testo
Continuiamo la discussione del protocollo HTTP.

\subsubsection{Tipi di richiesta HTTP}
Esistono più tipi di richiesta HTTP:
\begin{itemize}
	\item \textbf{GET:} richiede una risorsa dal server;
	\item \textbf{POST:} invia informazioni al server, ad esempio per trasferire un form.
	\item \textbf{HEAD:} richiede solo l'intestazione o \textit{header} della risorsa, ad esempio per controllare se ha già la versione più recente in cache;
	\item \textbf{PUT:} aggiorna o rimpiazza una risorsa ad un dato URL. Se non esiste, la crea;
	\item \textbf{DELETE:} Rimuove una risorsa a un dato URL;
	\item \textbf{CONNECT:} Stabilisce una connessione col server. Spesso è utilizzato per connessioni SSL (HTTPS);
	\item \textbf{TRACE:} Risponde con la stessa richiesta. Usata per motivi di debug, ad esempio dal programma \lstinline|traceroute| visto in 2.2.2;
	\item \textbf{OPTIONS:} Descrive le opzioni di comunicazione per la risorsa interessata. Utile per trovare quali metodi HTTP sono supportati dal server.
\end{itemize}

Più informazioni sulle specifiche del protocollo HTTP per sviluppatori web possono poi essere trovate in \url{https://raw.githubusercontent.com/seggiani-luca/appunti-web/481107c15776fac537b7882b6e0becfcb88b9886/master/master.pdf}. 

\subsubsection{Risposte HTTP}
Ad una richiesta HTTP su un server web alla porta 80 segue una \textbf{risposta}. Questa ha la seguente forma generale. Questa ha un header dalla seguente forma generale:
\begin{lstlisting}[language=html, style=codestyle]	
HTTP/1.1 200 OK
Date: Sun, 26 Sep 2010 20:09:20 GMT
Server: Apache/2.0.52 (CentOS)
Last-Modified: Tue, 30 Oct 2007 17:00:02 GMT
Content-Length: 2652
Keep-Alive: timeout=10, max=100
Connection: Keep-Alive
Content-Type: text/html
\end{lstlisting}
Anche queste righe sono separate da \lstinline|\r\n|, caratteri di ritorno carrello e nuova linea.

La prima riga definisce la \textbf{versione} protocollo, e la risposta del server in codice (200) e testo (OK).

Esiste un sistema di codici che comunicano le varie situazioni che si possono creare alla risposta.
Ad esempio, alcuni dei codici più tipici sono \textbf{200} (\textit{OK}), \textbf{301} (\textit{Moved Permanently}), \textbf{400} (\textit{Bad Request}), \textbf{404} (\textit{Not Found}) e \textbf{505} (\textit{Version not supported}).
Vediamo come i codici che iniziano da \textbf{100} riguardano \textit{messaggi informativi}, \textbf{200} condizioni di \textit{successo}, \textbf{300} di \textit{redirezione}, \textbf{400} errori \textit{client} e \textbf{500} errori \textit{server}.
Anche questo argomento viene approfondito all'URL riportato nella sezione precedente.

Seguono poi diverse coppie chiave-valore che contengono informazioni sul contenuto ottenuto, il server che l'ha fornito, ecc...

Possiamo rivolgere la nostra attenzione alle linee \lstinline|Keep-Alive| e \lstinline|Connection|: queste stabiliscono il tipo di connessione. Dall'esempio della scorsa lezione avevamo visto come una richiesta può esprimere la preferenza per connessioni \lstinline|keep-alive| (in HTTP persistente).
Qui il server ha consentito la connessione \lstinline|keep-alive|, stabilendo un timeout di 10 secondi e al massimo 100 richieste HTTP accettate.

\subsubsection{Meccanismo dei cookie}
Avevamo detto che il protocollo HTTP è \textit{stateless}, cioè privo di stato. Un modo per reintrodurre una qualche nozione di stato nel protocollo è adottare il meccanismo dei \textbf{cookie}.

Un cookie è una coppia chiave-valore, o semplicemente anche una sola chiave, che il server invia al client assieme ad una pagina per conservare informazione di stato.
Il browser che ha intenzione di preservare lo stato potrà a questo punto annettere il cookie ad ogni richiesta successiva, permettendo al server di "ricordare" quale utente sta lanciando la richiesta.

Prevediamo quindi una nuova linea di stato nell'header delle risposte HTTP (\lstinline|Set-Cookie:|) che dovrà contenere questi cookie, e una linea simile nell'header delle richieste HTTP (\lstinline|Cookie:|) che vengono dal browser.
A questo punto basterà mantenere un file (o volendo un database) sul browser utente che associa i cookie ad un dominio, e un database (qui obbligatoriamente, probabilmente anche massiccio) di \textit{backend} sul Web server che associa un cookie ad ogni utente.

Con il meccanismo dei cookie il protocollo stateless dell'HTTP diventa effettivamente \textit{stateful}, cioè capace di preservare lo stato fra più richieste (assunto che il browser sia disposto a reinviare ogni volta la stringa di cookie).

\subsection{Cache Web}
Per ridurre il tempo di accesso agli Web server (che possono trovarsi anche molto lontano dal browser) e ridurre il traffico sugli stessi, si può usare un sistema di \textit{caching}, cioè redirezionare il browser verso una \textit{cache Web} (o \textbf{server proxy}).

Un proxy duplica alcuni dei contenuti presenti sul server originale: se possiede la risorsa richiesta dall'utente, la restituisce, altrimenti contatta il server originale, se la procura e la fornisce all'utente. A questo punto mantiene la risorsa per richieste future (chiaramente gestendo la sua memoria, che è finita).

Chiaramente, i \textit{miss} dei proxy sono più lenti di un accesso diretto al server originale, ma di contro la soluzione diventa viabile e effettivamente utile quando si riesce a raggiungere una certa quota di \textit{hit}.

I proxy permettono a content provider meno distribuiti di distribuire contenuti più facilmente.
Solitamente sono installati dagli ISP (sia privati che istituzionali), con l'effetto collaterale (positivo) della riduzione del traffico sia sui server originali che sulla linea di accesso a Internet dell'ISP stesso.

Una nota interessante è che il proxy si comporta contemporaneamente sia come \textbf{client} che come \textbf{server}: dal browser è visto come \textit{server}, mentre dal server originale è visto come \textit{client}.

\subsubsection{GET condizionale}
Potremmo chiederci come fa li server proxy ad assicurarsi di mandarci sempre la copia più aggiornata della risorsa richiesta.

La soluzione è, lato server originale, non inviare la risorsa se è gia presente in cache: il server proxy può usare la linea di richiesta \lstinline|If-modified-since:| per specificare l'ultima versione che ha a disposizione, e il server può rispondere con una nuova copia (se esiste), o alternativamente con il codice \textbf{304} (\textit{Not Modified}).

Questo mantiene chiaramente un piccolo traffico in circolazione sul link, che però è comunque più piccolo del traffico richiesto per inviare risorse complete, e quindi non si traduce in carichi significativi per la rete.

\subsection{Protocollo HTTP/2}
L'obiettivo degli aggiornamenti del protocollo HTTP è principalmente quello di ridurre il ritardo in richieste HTTP multi oggetto.

\textbf{HTTP/1.1} ha introdotto richieste GET multiple eseguite in \textit{pipeline} su una sola connessione TCP.
Il server risponde \textit{in-order} usando l'algoritmo \textbf{FCFS} \textit(scheduling First Come First Served). 
Secondo questo algoritmo, gli oggetti più piccoli devono aspettare la trasmissione dietro gli oggetti più grandi (\textbf{HOL blocking}, \textit{Head Of Line blocking}).

Inoltre, il meccanismo di ritrasmissione dei segmenti TCP persi può rallentare ulteriormente le trasmissioni.

Potremmo pensare di risolvere i problemi di HTTP/1.1 usando altri algoritmi di scheduling:
\begin{itemize}
	\item Magari inviando prima gli oggetti più piccoli. Questo però ritarderebbe indefinitamente la trasmissione di oggetti più grandi, in quanto probabilmente ci sarà quasi sempre un'oggetto più piccolo;
	\item Usando un approccio \textbf{RR} (\textit{Round Robin}), dove gli oggetti più grandi vengono divisi in \textit{segmenti} più piccoli, e il server alterna fra i diversi client trasmettendo tali segmenti.
\end{itemize}

\textbf{HTTP/2} ha aumentato la flessibilità lato server nell'inviare oggetti al client.
In questo caso l'ordine di trasmissione degli oggetti è stabilito da codici di priorità inviati dai client, e da algoritmi più sofisticati lato server (come il RR visto prima).
Inoltre, il server può inoltrare (\textit{push}) ai client oggetti che non hanno richiesto.

In HTTP/2 restano comunque i problemi di stallo nel caso di ritrasmissioni di segmenti TCP persi: i browser sono invitati a mantenere più connessioni TCP parallele per ridurre gli stalli e aumentare il throughput complessivo.

Inoltre, non c'è sicurezza sopra il protocollo TCP.

\textbf{HTTP/3} ha introdotto meccanismi di sicurezza, e controllo di errori per oggetto e congestioni su UDP. Approfondiremo questo aspetto quando parleremo del livello transport.


\end{document}

\end{document}