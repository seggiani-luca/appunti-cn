\documentclass[a4paper,11pt]{article}
\usepackage[a4paper, margin=8em]{geometry}

% usa i pacchetti per la scrittura in italiano
\usepackage[french,italian]{babel}
\usepackage[T1]{fontenc}
\usepackage[utf8]{inputenc}
\frenchspacing 

% usa i pacchetti per la formattazione matematica
\usepackage{amsmath, amssymb, amsthm, amsfonts}

% usa altri pacchetti
\usepackage{gensymb}
\usepackage{hyperref}
\usepackage{standalone}

\usepackage{colortbl}

\usepackage{xstring}
\usepackage{karnaugh-map}

% imposta il titolo
\title{Appunti Reti Informatiche}
\author{Luca Seggiani}
\date{2025}

% imposta lo stile
% usa helvetica
\usepackage[scaled]{helvet}
% usa palatino
\usepackage{palatino}
% usa un font monospazio guardabile
\usepackage{lmodern}

\renewcommand{\rmdefault}{ppl}
\renewcommand{\sfdefault}{phv}
\renewcommand{\ttdefault}{lmtt}

% circuiti
\usepackage{circuitikz}
\usetikzlibrary{babel}

% testo cerchiato
\newcommand*\circled[1]{\tikz[baseline=(char.base)]{
            \node[shape=circle,draw,inner sep=2pt] (char) {#1};}}

% disponi il titolo
\makeatletter
\renewcommand{\maketitle} {
	\begin{center} 
		\begin{minipage}[t]{.8\textwidth}
			\textsf{\huge\bfseries \@title} 
		\end{minipage}%
		\begin{minipage}[t]{.2\textwidth}
			\raggedleft \vspace{-1.65em}
			\textsf{\small \@author} \vfill
			\textsf{\small \@date}
		\end{minipage}
		\par
	\end{center}

	\thispagestyle{empty}
	\pagestyle{fancy}
}
\makeatother

% disponi teoremi
\usepackage{tcolorbox}
\newtcolorbox[auto counter, number within=section]{theorem}[2][]{%
	colback=blue!10, 
	colframe=blue!40!black, 
	sharp corners=northwest,
	fonttitle=\sffamily\bfseries, 
	title=Teorema~\thetcbcounter: #2, 
	#1
}

% disponi definizioni
\newtcolorbox[auto counter, number within=section]{definition}[2][]{%
	colback=red!10,
	colframe=red!40!black,
	sharp corners=northwest,
	fonttitle=\sffamily\bfseries,
	title=Definizione~\thetcbcounter: #2,
	#1
}

% disponi codice
\usepackage{listings}
\usepackage[table]{xcolor}

\definecolor{codegreen}{rgb}{0,0.6,0}
\definecolor{codegray}{rgb}{0.5,0.5,0.5}
\definecolor{codepurple}{rgb}{0.58,0,0.82}
\definecolor{backcolour}{rgb}{0.95,0.95,0.92}

\lstdefinestyle{codestyle}{
		backgroundcolor=\color{black!5}, 
		commentstyle=\color{codegreen},
		keywordstyle=\bfseries\color{magenta},
		numberstyle=\sffamily\tiny\color{black!60},
		stringstyle=\color{green!50!black},
		basicstyle=\ttfamily\footnotesize,
		breakatwhitespace=false,         
		breaklines=true,                 
		captionpos=b,                    
		keepspaces=true,                 
		numbers=left,                    
		numbersep=5pt,                  
		showspaces=false,                
		showstringspaces=false,
		showtabs=false,                  
		tabsize=2
}

\lstdefinestyle{shellstyle}{
		backgroundcolor=\color{black!5}, 
		basicstyle=\ttfamily\footnotesize\color{black}, 
		commentstyle=\color{black}, 
		keywordstyle=\color{black},
		numberstyle=\color{black!5},
		stringstyle=\color{black}, 
		showspaces=false,
		showstringspaces=false, 
		showtabs=false, 
		tabsize=2, 
		numbers=none, 
		breaklines=true
}


\lstdefinelanguage{assembler}{ 
  keywords={AAA, AAD, AAM, AAS, ADC, ADCB, ADCW, ADCL, ADD, ADDB, ADDW, ADDL, AND, ANDB, ANDW, ANDL,
        ARPL, BOUND, BSF, BSFL, BSFW, BSR, BSRL, BSRW, BSWAP, BT, BTC, BTCB, BTCW, BTCL, BTR, 
        BTRB, BTRW, BTRL, BTS, BTSB, BTSW, BTSL, CALL, CBW, CDQ, CLC, CLD, CLI, CLTS, CMC, CMP,
        CMPB, CMPW, CMPL, CMPS, CMPSB, CMPSD, CMPSW, CMPXCHG, CMPXCHGB, CMPXCHGW, CMPXCHGL,
        CMPXCHG8B, CPUID, CWDE, DAA, DAS, DEC, DECB, DECW, DECL, DIV, DIVB, DIVW, DIVL, ENTER,
        HLT, IDIV, IDIVB, IDIVW, IDIVL, IMUL, IMULB, IMULW, IMULL, IN, INB, INW, INL, INC, INCB,
        INCW, INCL, INS, INSB, INSD, INSW, INT, INT3, INTO, INVD, INVLPG, IRET, IRETD, JA, JAE,
        JB, JBE, JC, JCXZ, JE, JECXZ, JG, JGE, JL, JLE, JMP, JNA, JNAE, JNB, JNBE, JNC, JNE, JNG,
        JNGE, JNL, JNLE, JNO, JNP, JNS, JNZ, JO, JP, JPE, JPO, JS, JZ, LAHF, LAR, LCALL, LDS,
        LEA, LEAVE, LES, LFS, LGDT, LGS, LIDT, LMSW, LOCK, LODSB, LODSD, LODSW, LOOP, LOOPE,
        LOOPNE, LSL, LSS, LTR, MOV, MOVB, MOVW, MOVL, MOVSB, MOVSD, MOVSW, MOVSX, MOVSXB,
        MOVSXW, MOVSXL, MOVZX, MOVZXB, MOVZXW, MOVZXL, MUL, MULB, MULW, MULL, NEG, NEGB, NEGW,
        NEGL, NOP, NOT, NOTB, NOTW, NOTL, OR, ORB, ORW, ORL, OUT, OUTB, OUTW, OUTL, OUTSB, OUTSD,
        OUTSW, POP, POPL, POPW, POPB, POPA, POPAD, POPF, POPFD, PUSH, PUSHL, PUSHW, PUSHB, PUSHA, 
				PUSHAD, PUSHF, PUSHFD, RCL, RCLB, RCLW, MOVSL, MOVSB, MOVSW, STOSL, STOSB, STOSW, LODSB, LODSW,
				LODSL, INSB, INSW, INSL, OUTSB, OUTSL, OUTSW
        RCLL, RCR, RCRB, RCRW, RCRL, RDMSR, RDPMC, RDTSC, REP, REPE, REPNE, RET, ROL, ROLB, ROLW,
        ROLL, ROR, RORB, RORW, RORL, SAHF, SAL, SALB, SALW, SALL, SAR, SARB, SARW, SARL, SBB,
        SBBB, SBBW, SBBL, SCASB, SCASD, SCASW, SETA, SETAE, SETB, SETBE, SETC, SETE, SETG, SETGE,
        SETL, SETLE, SETNA, SETNAE, SETNB, SETNBE, SETNC, SETNE, SETNG, SETNGE, SETNL, SETNLE,
        SETNO, SETNP, SETNS, SETNZ, SETO, SETP, SETPE, SETPO, SETS, SETZ, SGDT, SHL, SHLB, SHLW,
        SHLL, SHLD, SHR, SHRB, SHRW, SHRL, SHRD, SIDT, SLDT, SMSW, STC, STD, STI, STOSB, STOSD,
        STOSW, STR, SUB, SUBB, SUBW, SUBL, TEST, TESTB, TESTW, TESTL, VERR, VERW, WAIT, WBINVD,
        XADD, XADDB, XADDW, XADDL, XCHG, XCHGB, XCHGW, XCHGL, XLAT, XLATB, XOR, XORB, XORW, XORL},
  keywordstyle=\color{blue}\bfseries,
  ndkeywordstyle=\color{darkgray}\bfseries,
  identifierstyle=\color{black},
  sensitive=false,
  comment=[l]{\#},
  morecomment=[s]{/*}{*/},
  commentstyle=\color{purple}\ttfamily,
  stringstyle=\color{red}\ttfamily,
  morestring=[b]',
  morestring=[b]"
}

\lstset{language=assembler, style=codestyle}

% disponi sezioni
\usepackage{titlesec}

\titleformat{\section}
	{\sffamily\Large\bfseries} 
	{\thesection}{1em}{} 
\titleformat{\subsection}
	{\sffamily\large\bfseries}   
	{\thesubsection}{1em}{} 
\titleformat{\subsubsection}
	{\sffamily\normalsize\bfseries} 
	{\thesubsubsection}{1em}{}

% tikz
\usepackage{tikz}

% float
\usepackage{float}

% grafici
\usepackage{pgfplots}
\pgfplotsset{width=10cm,compat=1.9}

% disponi alberi
\usepackage{forest}

\forestset{
	rectstyle/.style={
		for tree={rectangle,draw,font=\large\sffamily}
	},
	roundstyle/.style={
		for tree={circle,draw,font=\large}
	}
}

% disponi algoritmi
\usepackage{algorithm}
\usepackage{algorithmic}
\makeatletter
\renewcommand{\ALG@name}{Algoritmo}
\makeatother

% disponi numeri di pagina
\usepackage{fancyhdr}
\fancyhf{} 
\fancyfoot[L]{\sffamily{\thepage}}

\makeatletter
\fancyhead[L]{\raisebox{1ex}[0pt][0pt]{\sffamily{\@title \ \@date}}} 
\fancyhead[R]{\raisebox{1ex}[0pt][0pt]{\sffamily{\@author}}}
\makeatother

\begin{document}
% sezione (data)
\section{Lezione del 05-12-25}

% stili pagina
\thispagestyle{empty}
\pagestyle{fancy}

% testo
Continuiamo a trattare la famiglia di tecnologie IEEE 802.11.

\subsubsection{Mobilità in 802.11}
Iniziamo a vedere come le tecnologie 802.11 implementano la \textbf{mobilità}.
Ciò che vogliamo è che, per un host che si sposta in una rete con più AP (collegati a diversi switch o almeno a diverse porte fisiche di switch), il subnet IP di appartenenza resti costante.

Ricordiamo di aver parlato delle capabilità di \textit{self-learning} della rete: gli switch \textit{"apprendono"} che un host è raggiungibile ad una certa porta fisica, e quindi inoltrano pacchetti destinati a tale host attraverso quella porta.
Ricordiamo anche che le entrate nella tabella di switching hanno time-to-live, per cui verranno resettate dopo un certo periodo di tempo, rimettendo l'host effettivamente sulla rete.

\subsubsection{Funzionalità avanzate di 802.11}
802.11 prevede anche alcune funzionalità più \textbf{avanzate}, fra cui notiamo:
\begin{itemize}
	\item \textbf{Adattamento di bitrate}: AP e nodi mobili possono cambiare la loro frequenza di trasmissione in maniera dinamica, cambiando le loro tecniche di modulazione a livello fisico. Queste, in praticolare, sono \textit{QAM256} a 8 Mbps, \textit{QAM16} a 4 Mbps e \textit{BPSK} a 1Mbps, con diverse curve di correlazione SNR\/BER:
		\begin{center}
			\includegraphics[scale=0.38]{../figures/80211_phys.png}
		\end{center}

		Notiamo che ciò che l'utente finale vorrebbe è chiaramente il massimo bitrate possibile, ma questo non è sempre possibile in quanto non solo si possono avere caratteristiche di rete più o meno sfavorevoli, ma tali caratteristiche si comportano in maniera non esattamente intuitiva (diverse tecnologie hanno fasce "ottime" di trasmissione, il raggio di trasmissione non è esattamente sferico, ecc...).
		
		Si sceglie quindi di valutare la qualità della trasmissione in tempo reale, e variare adattivamente il protocollo di livello fisico (e quindi il bitrate), partendo da quello con bitrate maggiore.
		Nel dettaglio:
		\begin{enumerate}
			\item Si parte con il protocollo di bitrate massimo (diciamo QAM256);
			\item Si valuta il BER: mentre il dispositivo si allontana dall'AP, il SNR diminuisce e il BER aumenta;
			\item Quando il BER supera una certa soglia si passa al protocollo di bitrate immediatamente minore (diciamo QAM16). In questo modo il BER diminuisce, e il processo può essere iterato per raggiungere il protocollo ottimo date le caratteristiche di rete. 
		\end{enumerate}
	\item \textbf{Risparmio energetico}: a partire dai primi anni '90 si sono iniziate a sviluppare tecnologie che avevano bisogno di effettuare \textit{risparmio energetico} (dispositivi portatili a batteria, ecc...). Per questo 802.11 prevede alcune funzionalità che supportano tali necessità.

		I nodi possono inviare frame che informano l'AP che hanno intenzione di \textit{"dormire"} fino al successivo frame di beaconing.
		A questo punto:
		\begin{itemize}
			\item L'AP saprà di non dover trasmettere frame a questi nodi, e si occuperà di bufferizzarli;
			\item Il nodo andrà in risparmio energetico fino al successivo frame di beaconing (che verrà ricevuto dalla NIC secondo interruzione, e quindi sveglierà il dispositivo anche se questo era in stato di HALT). 
		\end{itemize}

		Quando l'AP preparerà un frame di beaconing, si comporterà come segue:
		\begin{itemize}
			\item Includerà nel frame una lista di nodi associati ai frame per tali nodi che sono stati bufferizzati e devono essere inviati;
			\item I nodi riceveranno tale frame di beacnoing. Quindi resteranno svegli per ricevere tali frame, nel caso ne abbiano, o torneranno in risparmio energetico in attesa di un altro frame di beaconing, e così via. 
		\end{itemize}
\end{itemize}

\subsection{PAN}
Se con lo standard IEEE 802.11 abbiamo studiato le reti LAN wireless, adesso vogliamo studiare le reti \textbf{PAN} (\textit{Personal Area Network}). Come tecnologia di esempio, in questo caso, usiamo \textit{Bluetooth}.

Bluetooth è una tecnologia di trasmissione wireless a piccolo raggio di copertura, inizialmente pensata per rimpiazzare i cavi come mezzo di trasmissione fra dispositivi. 
Si basa su TDM (Visto in 12.3.1), a slot da 625 microsecondi. 

\subsubsection{Frequency hopping}
Bluetooh sfrutta il \textbf{frequency hopping}: il trasmettitore sfrutta 79 canali di frequenze, cambiando frequenza da slot a slot in un ordine pseudo casuale noto sia a ricevitori che a trasmettitori.

\subsubsection{Risparmio energetico in Bluetooth}
Il \textbf{risparmio energetico} è parte integrante di Bluetooth: è prevista infatti una funzionalità di \textit{parking}, dove un dispositivo può \textit{"parcheggiarsi"} (un procedimento simile a quello dell'\textit{"addormentarsi"} dei dispositivi WiFi), e svegliarsi in seguito per risparmiare batteria.

\subsubsection{Bootstrapping in Bluetooth}
Bluetooth è pensato per essere \textit{self-starting}, e infatti prevede un processo di \textbf{bootstrapping} che porta alla formazione di piccole reti ad hoc formate da dispositivi.

\subsection{Reti cellulari}
Le \textbf{reti cellulari}, inizialmente pensate per la trasmissione telefonica, sono oggi reti globali usate per l'accesso ad internet su lunge distanze. 

Esistono diverse famiglie di standard, con nomi noti al pubblico (3G, 4G, 5G), sviluppati da consorzi di compagnie del settore.

Vediamo quindi le similarità principali fra l'Internet cablato e quello mobile:
\begin{itemize}
	\item Esiste una distinzione fra edge e core della rete, ma entrambi appartengono allo stesso operatore;
	\item Si parla di rete cellulare globale: una rete di reti;
	\item C'è un uso diffuso dei protocolli studiati: HTTP, DNS, TCP, UDP, IP, NAT, separazione dei piani dati/controllo, SDN, Ethernet, tunneling, ecc...;
	\item La rete mobile è interconnessa a Internet cablato;
\end{itemize}
e quindi le differenze:
\begin{itemize}
	\item Il livello di collegamento wireless differente;
	\item La mobilità è visto come un servizio di prima classe;
	\item Esiste un “identità” dell’utente (tramite SIM);
	\item Cambia il modello di business: gli utenti si abbonano a un operatore cellulare;
	\item C'è una forte nozione di “rete domestica” rispetto al roaming in reti visitate;
	\item Esiste accesso globale, ma con infrastruttura di autenticazione e accordi di interconnessione tra operatori.
	\end{itemize}

\subsubsection{Elementi dell'architettura 4G}
Vediamo quindi la struttura di una rete mobile, prendendo come esempio lo standard \textbf{4G}.

Prevediamo una \textbf{base station} (\textit{eNode-B}), non dissimile dagli AP WiFi ma su scala vastamente piià grande.
La BS è quindi connessa ad una rete di router che va a formare l'infrastruttura messa a disposizione della rete dalla compagnia che gestisce la rete mobile.
In tale rete prevediamo almeno due gateway:
\begin{itemize}
	\item Un \textbf{serving gateway} (\textit{S-GW}), che si occupa di servire i singoli utenti della rete, sfruttando almeno 2 database:
		\begin{itemize}
			\item Il \textbf{MME} (\textit{Mobility Management Entity}), che gestisce la mobilità sulla rete;
			\item L'\textbf{HSS} (\textit{Home Subcriber Service}), che si occupa di gestire l'autenticazione degli utenti alla rete. 
		\end{itemize}
	\item Un \textbf{PDN gateway} (\textit{P-GW}), che si occupa di fornire l'accesso all'Internet vero e proprio.
\end{itemize}

Fra S-GW e P-GW sono previsti diversi protocollo ad-hoc, che al livello \textit{data plane} implementano tunneling IP: l'obiettivo è quello di usare protocolli nuovi al livello fisico e link, e i buon vecchi protocolli livello transport e application a cui siamo abituati per facilitare la mobilità. 

# dice l'anastasi fallo te

\end{document}
