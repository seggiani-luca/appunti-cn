\documentclass[a4paper,11pt]{article}
\usepackage[a4paper, margin=8em]{geometry}

% usa i pacchetti per la scrittura in italiano
\usepackage[french,italian]{babel}
\usepackage[T1]{fontenc}
\usepackage[utf8]{inputenc}
\frenchspacing 

% usa i pacchetti per la formattazione matematica
\usepackage{amsmath, amssymb, amsthm, amsfonts}

% usa altri pacchetti
\usepackage{gensymb}
\usepackage{hyperref}
\usepackage{standalone}

\usepackage{colortbl}

\usepackage{xstring}
\usepackage{karnaugh-map}

% imposta il titolo
\title{Appunti Reti Informatiche}
\author{Luca Seggiani}
\date{2025}

% imposta lo stile
% usa helvetica
\usepackage[scaled]{helvet}
% usa palatino
\usepackage{palatino}
% usa un font monospazio guardabile
\usepackage{lmodern}

\renewcommand{\rmdefault}{ppl}
\renewcommand{\sfdefault}{phv}
\renewcommand{\ttdefault}{lmtt}

% circuiti
\usepackage{circuitikz}
\usetikzlibrary{babel}

% testo cerchiato
\newcommand*\circled[1]{\tikz[baseline=(char.base)]{
            \node[shape=circle,draw,inner sep=2pt] (char) {#1};}}

% disponi il titolo
\makeatletter
\renewcommand{\maketitle} {
	\begin{center} 
		\begin{minipage}[t]{.8\textwidth}
			\textsf{\huge\bfseries \@title} 
		\end{minipage}%
		\begin{minipage}[t]{.2\textwidth}
			\raggedleft \vspace{-1.65em}
			\textsf{\small \@author} \vfill
			\textsf{\small \@date}
		\end{minipage}
		\par
	\end{center}

	\thispagestyle{empty}
	\pagestyle{fancy}
}
\makeatother

% disponi teoremi
\usepackage{tcolorbox}
\newtcolorbox[auto counter, number within=section]{theorem}[2][]{%
	colback=blue!10, 
	colframe=blue!40!black, 
	sharp corners=northwest,
	fonttitle=\sffamily\bfseries, 
	title=Teorema~\thetcbcounter: #2, 
	#1
}

% disponi definizioni
\newtcolorbox[auto counter, number within=section]{definition}[2][]{%
	colback=red!10,
	colframe=red!40!black,
	sharp corners=northwest,
	fonttitle=\sffamily\bfseries,
	title=Definizione~\thetcbcounter: #2,
	#1
}

% disponi codice
\usepackage{listings}
\usepackage[table]{xcolor}

\definecolor{codegreen}{rgb}{0,0.6,0}
\definecolor{codegray}{rgb}{0.5,0.5,0.5}
\definecolor{codepurple}{rgb}{0.58,0,0.82}
\definecolor{backcolour}{rgb}{0.95,0.95,0.92}

\lstdefinestyle{codestyle}{
		backgroundcolor=\color{black!5}, 
		commentstyle=\color{codegreen},
		keywordstyle=\bfseries\color{magenta},
		numberstyle=\sffamily\tiny\color{black!60},
		stringstyle=\color{green!50!black},
		basicstyle=\ttfamily\footnotesize,
		breakatwhitespace=false,         
		breaklines=true,                 
		captionpos=b,                    
		keepspaces=true,                 
		numbers=left,                    
		numbersep=5pt,                  
		showspaces=false,                
		showstringspaces=false,
		showtabs=false,                  
		tabsize=2
}

\lstdefinestyle{shellstyle}{
		backgroundcolor=\color{black!5}, 
		basicstyle=\ttfamily\footnotesize\color{black}, 
		commentstyle=\color{black}, 
		keywordstyle=\color{black},
		numberstyle=\color{black!5},
		stringstyle=\color{black}, 
		showspaces=false,
		showstringspaces=false, 
		showtabs=false, 
		tabsize=2, 
		numbers=none, 
		breaklines=true
}


\lstdefinelanguage{assembler}{ 
  keywords={AAA, AAD, AAM, AAS, ADC, ADCB, ADCW, ADCL, ADD, ADDB, ADDW, ADDL, AND, ANDB, ANDW, ANDL,
        ARPL, BOUND, BSF, BSFL, BSFW, BSR, BSRL, BSRW, BSWAP, BT, BTC, BTCB, BTCW, BTCL, BTR, 
        BTRB, BTRW, BTRL, BTS, BTSB, BTSW, BTSL, CALL, CBW, CDQ, CLC, CLD, CLI, CLTS, CMC, CMP,
        CMPB, CMPW, CMPL, CMPS, CMPSB, CMPSD, CMPSW, CMPXCHG, CMPXCHGB, CMPXCHGW, CMPXCHGL,
        CMPXCHG8B, CPUID, CWDE, DAA, DAS, DEC, DECB, DECW, DECL, DIV, DIVB, DIVW, DIVL, ENTER,
        HLT, IDIV, IDIVB, IDIVW, IDIVL, IMUL, IMULB, IMULW, IMULL, IN, INB, INW, INL, INC, INCB,
        INCW, INCL, INS, INSB, INSD, INSW, INT, INT3, INTO, INVD, INVLPG, IRET, IRETD, JA, JAE,
        JB, JBE, JC, JCXZ, JE, JECXZ, JG, JGE, JL, JLE, JMP, JNA, JNAE, JNB, JNBE, JNC, JNE, JNG,
        JNGE, JNL, JNLE, JNO, JNP, JNS, JNZ, JO, JP, JPE, JPO, JS, JZ, LAHF, LAR, LCALL, LDS,
        LEA, LEAVE, LES, LFS, LGDT, LGS, LIDT, LMSW, LOCK, LODSB, LODSD, LODSW, LOOP, LOOPE,
        LOOPNE, LSL, LSS, LTR, MOV, MOVB, MOVW, MOVL, MOVSB, MOVSD, MOVSW, MOVSX, MOVSXB,
        MOVSXW, MOVSXL, MOVZX, MOVZXB, MOVZXW, MOVZXL, MUL, MULB, MULW, MULL, NEG, NEGB, NEGW,
        NEGL, NOP, NOT, NOTB, NOTW, NOTL, OR, ORB, ORW, ORL, OUT, OUTB, OUTW, OUTL, OUTSB, OUTSD,
        OUTSW, POP, POPL, POPW, POPB, POPA, POPAD, POPF, POPFD, PUSH, PUSHL, PUSHW, PUSHB, PUSHA, 
				PUSHAD, PUSHF, PUSHFD, RCL, RCLB, RCLW, MOVSL, MOVSB, MOVSW, STOSL, STOSB, STOSW, LODSB, LODSW,
				LODSL, INSB, INSW, INSL, OUTSB, OUTSL, OUTSW
        RCLL, RCR, RCRB, RCRW, RCRL, RDMSR, RDPMC, RDTSC, REP, REPE, REPNE, RET, ROL, ROLB, ROLW,
        ROLL, ROR, RORB, RORW, RORL, SAHF, SAL, SALB, SALW, SALL, SAR, SARB, SARW, SARL, SBB,
        SBBB, SBBW, SBBL, SCASB, SCASD, SCASW, SETA, SETAE, SETB, SETBE, SETC, SETE, SETG, SETGE,
        SETL, SETLE, SETNA, SETNAE, SETNB, SETNBE, SETNC, SETNE, SETNG, SETNGE, SETNL, SETNLE,
        SETNO, SETNP, SETNS, SETNZ, SETO, SETP, SETPE, SETPO, SETS, SETZ, SGDT, SHL, SHLB, SHLW,
        SHLL, SHLD, SHR, SHRB, SHRW, SHRL, SHRD, SIDT, SLDT, SMSW, STC, STD, STI, STOSB, STOSD,
        STOSW, STR, SUB, SUBB, SUBW, SUBL, TEST, TESTB, TESTW, TESTL, VERR, VERW, WAIT, WBINVD,
        XADD, XADDB, XADDW, XADDL, XCHG, XCHGB, XCHGW, XCHGL, XLAT, XLATB, XOR, XORB, XORW, XORL},
  keywordstyle=\color{blue}\bfseries,
  ndkeywordstyle=\color{darkgray}\bfseries,
  identifierstyle=\color{black},
  sensitive=false,
  comment=[l]{\#},
  morecomment=[s]{/*}{*/},
  commentstyle=\color{purple}\ttfamily,
  stringstyle=\color{red}\ttfamily,
  morestring=[b]',
  morestring=[b]"
}

\lstset{language=assembler, style=codestyle}

% disponi sezioni
\usepackage{titlesec}

\titleformat{\section}
	{\sffamily\Large\bfseries} 
	{\thesection}{1em}{} 
\titleformat{\subsection}
	{\sffamily\large\bfseries}   
	{\thesubsection}{1em}{} 
\titleformat{\subsubsection}
	{\sffamily\normalsize\bfseries} 
	{\thesubsubsection}{1em}{}

% tikz
\usepackage{tikz}

% float
\usepackage{float}

% grafici
\usepackage{pgfplots}
\pgfplotsset{width=10cm,compat=1.9}

% disponi alberi
\usepackage{forest}

\forestset{
	rectstyle/.style={
		for tree={rectangle,draw,font=\large\sffamily}
	},
	roundstyle/.style={
		for tree={circle,draw,font=\large}
	}
}

% disponi algoritmi
\usepackage{algorithm}
\usepackage{algorithmic}
\makeatletter
\renewcommand{\ALG@name}{Algoritmo}
\makeatother

% disponi numeri di pagina
\usepackage{fancyhdr}
\fancyhf{} 
\fancyfoot[L]{\sffamily{\thepage}}

\makeatletter
\fancyhead[L]{\raisebox{1ex}[0pt][0pt]{\sffamily{\@title \ \@date}}} 
\fancyhead[R]{\raisebox{1ex}[0pt][0pt]{\sffamily{\@author}}}
\makeatother

\begin{document}
% sezione (data)
\section{Lezione del 17-10-25}

% stili pagina
\thispagestyle{empty}
\pagestyle{fancy}

% testo

\subsubsection{Pseudocodice del selective repeat}
Continuiamo la discussione del selective repeat.
Per farlo, non usiamo più un approccio FSM, ma descriviamolo in \textit{pseudocodice}.

Vediamo allora il trasmettitore:
\begin{center}
\begin{algorithm}
	\caption{Trasmettitore \textit{selective repeat}}
\begin{algorithmic}
	\IF{ci sono dati dall'alto}
		\STATE Se il prossimo numero di sequenza è nella finestra, invia il pacchetto
	\ENDIF
	\IF{timeout$(n)$}
		\STATE Reinvia il pacchetto $n$ e riavvia il timer
	\ENDIF
	\IF{ACK$(n)$ nella finestra}
		\STATE Marchia il pacchetto $n$ come \textit{ricevuto}
		\STATE Se era il pacchetto non ricevuto con numero di sequenza minore, avanza la finestra fino al prossimo pacchetto non ricevuto
	\ENDIF
\end{algorithmic}
\end{algorithm}
\end{center}

\newpage

e quindi il ricevitore:
\begin{center}
\begin{algorithm}
	\caption{Ricevitore \textit{selective repeat}}
\begin{algorithmic}
	\IF{ricevuto pacchetto $n$ in finestra}
		\STATE invia ACK$(n)$
		\STATE se è fuori ordine, bufferizzalo
		\STATE se è in ordine, consegna il pacchetto (e quindi i successivi bufferizzati), e avanza la finestra al prossimo pacchetto non ancora ricevuto
	\ENDIF
	\IF{ricevuto pacchetto $n$ prima della finestra}
		\STATE invia ACK$(n)$
	\ENDIF
	\IF{ricevuto pacchetto $n$ dopo la finestra}
		\STATE ignoralo
	\ENDIF
\end{algorithmic}
\end{algorithm}
\end{center}

Notiamo quindi che sia il trasmettitore che il ricevitore mantengono le loro finestre, il primo per i pacchetti marchiati come da inviare, il secondo per i pacchetti che può bufferizzare.

\subsection{Protocollo PPP}
Vediamo quindi un protocollo reale di livello \textit{datalink} che implementa le funzionalità che abbiamo descritto finora.
Questo è il protocollo \textbf{PPP} (\textit{Point-to-Point Protocol}), e viene oggi usato dai router per parlare direttamente l'uno con l'altro.

Le funzionalità del PPP sono:
\begin{itemize}
	\item \textbf{Framing} pacchetti: provvede all'incapsulamento dei datagrammi livello network in frame livello datalink. Può portare dati livello network di qualsiasi protocollo di rete (non solo IP);
	\item \textbf{Trasparenza bit}: è capace di portare qualsiasi pattern di bit nel campo dati;
	\item \textbf{Rilevamento errori}: c'è ma non c'è correzione;
	\item \textbf{Verifica connessione}: permette di verificare e segnalare il fallimento del link al livello network;
	\item \textbf{Negoziamento indirizzi} al livello network: gli host possono imparare e configurare gli indirizzi di rete altrui.
\end{itemize}

Vediamo che non è presente quanto abbiamo detto sul \textit{trasferimento affidabile} di dati: questo lo rende effettivamente un link \textbf{inaffidabile} (abbiamo visto che questa funzionalità viene reintrodotta a livello \textit{transport}).

In particolare, PPP non è provvisto di:
\begin{itemize}
	\item Correzione errori;
	\item Recupero da errori;
	\item Controllo di flusso;
	\item Consegna fuori ordine;
	\item Comunicazioni \textit{point-multipoint}. 
\end{itemize}

Ci si aspetta invece che queste funzionalità vengano delegate a protocolli di livello superiore come \textit{TCP}.

\subsubsection{Pacchetto PPP}
Iniziamo a vedere la struttura di un pacchetto PPP:
\begin{lstlisting}[style=codestyle]	
1 byte		1 byte		1 byte		1 byte		 variabile	2 o 4 byte	1 byte
<flag>		<address>	<control>	<protocol> <info>			<check>			<flag>
\end{lstlisting}

\begin{itemize}
	\item I campi \textbf{flag} fanno da delimitatori per il framing;
	\item Il campo \textbf{address} è effettivamente inutile: il PPP non permette comunicazine \textit{point-multipoint} (più destinatari) e quindi è sempre impostato tutto a 1 (l'indirizzo di \textit{broadcast});
	\item Il campo \textbf{control} è ugualmente inutile;
	\item Il campo \textbf{protocol} stabilisce il protocollo di livello superiore da usare (ad esempio TCP);
	\item Il campo \textbf{info} contiene i dati veri e propri, ed è a dimensione variabile da 0 a 1500 byte (la dimensione massima può essere negoziata);
	\item Il campo \textbf{check} supporta il CRC per il rilevamento errori.
\end{itemize}

Notiamo che garantire la \textit{trasparenza bit} non è immediato: se si spedisse un campo info che contiene una copia esatta del marcatore di flag terminale che ci aspettiamo, come dovrebbe fare il ricevitore a capire che tale campo va spedito e non è effettivamente il terminatore?

Risolviamo il problema sfruttando il \textbf{byte stuffing/unstuffing}, meccanismo per noi sostanzialmente analogo a quello delle \textit{sequenze di escape}.
Si antepone quindi al byte problematico un'altro specifico byte, di escape, che il ricevitore potrà poi interpretare come segnalatore (di prendere \textit{"alla lettera"} il byte sucessivo), e quindi buttare via.

\subsection{Link multipli}
Veniamo ora a descrivere le reti formate da più \textbf{link di accesso}.
La prima distinzione che dobbiamo fare è fra i due tipi di link che possiamo incontrare:
\begin{itemize}
	\item Link \textbf{punto-punto}: sono del tipo che abbiamo visto finora, e collegano solo due dispositivi fra di loro;
	\item Link \textbf{broadcast} (detti anche a \textit{mezzo condiviso}): sono tipici di tecnologie come il \textit{vecchio Ethernet}, le reti mobili e reti wireless come 802.11 (\textit{WiFi}). In questo tipo di rete i dispositivi possono parlare con tutti gli altri dispositivi, contemporaneamente.  
\end{itemize}

Il problema delle reti broadcast è chiaramente l'\textbf{interferenza}: si possono verificare \textit{collisioni} se un nodo riceve 2 o più segnali contemporaneamente.

\subsubsection{Protocolli MAC}
Dobbiamo quindi definire un protocollo di accesso multiplo, cioè un \textbf{algoritmo distribuito} che determini come i nodi condividono il canale (decida quale nodo può trasettere).
Il problema è che tale protocollo deve sfruttare comunicazione sul canale condiviso stesso, in quanto non abbiamo sempre a disposizione un altro canale \textit{fuori banda} da usare per la coordinazione delle comunicazioni. 

Quello che desideriamo è, dato un canale ad accesso multiplo \textbf{MAC} (\textit{Multiple Access Channel}) con capacità di $R$ bit al secondo, che:
\begin{enumerate}
	\item Quando un nodo vuole trasmettere, può farlo a capacità $R$;
	\item Quando $M$ nodi vogliono trasmettere, questi possono farlo a capacità $R/M$;
	\item Il sistema sia completamente decentralizzato: non ci siano nodi speciali che coordinano le comunicazioni, né clock di sincronizzazione;
	\item Il sistema sia \textit{semplice}.
\end{enumerate}

Chiameremo i protocolli che ci permettono di fare ciò protocolli \textbf{MAC} (acronimo già visto).

\subsubsection{Tassonomia dei MAC}
Esistono tre classi principali di protocolli MAC:
\begin{enumerate}
	\item \textbf{Partizionamento canali}: si divide il canale in \textit{"pezzi"} (slot temporali, bande di frequenza), e si alloca ogni \textit{"pezzo"} all'uso esclusivo di un nodo.
		Un idea interessante è quella della sovrapposizione di \textbf{codice}: in questo caso ogni nodo parla usando una codifica particolare, e si riesce a distinguere fra nodi rilevanto tali codifiche;
	\item \textbf{Accesso casuale}: il canale non viene diviso e si permettono le collisioni. In questo caso chiaramente l'approccio alla trasmissione sarà di:
		\begin{itemize}
			\item Rilevare quando il canale è libero;
			\item Aspettare un quanto temporale (sperabilmente piccolo e casuale, in modo da ridurre le collisioni);
			\item Iniziare a trasmettere.
		\end{itemize}

		In questo caso chiaramente bisogna dotarsi di un sistema per \textit{recuperare} le collisioni, che inevitabilmente prima o poi accadranno;
	\item \textbf{Turni}: i nodi si dividono a turni, ma i nodi con più dati da inviare possono prendere turni più lunghi.
\end{enumerate}

\subsubsection{Protocollo TDMA}
Vediamo il primo protocollo MAC, il \textbf{TDMA} (\textit{Time Division Multiple Access}). Questo fa parte della categoria (1) dei protocolli MAC, e quindi è a \textit{partizionamento canali}. 

Si fornisce l'accesso al canale di accesso in \textit{round}.
Ogni stazione otiene uno slot di accesso a lunghezza fissa in ogni round.
Gli slot non utilizzati tengono il canale a riposo.

Con questo protocollo riusciamo a soddisfare i primi 2 requisiti che ci eravamo dati. La decentralizzazione (requisito 3) non è immediata, in quanto bisogna capire quando i round iniziano e l'ordine di comunicazione ad ogni round.
Immaginiamo che i nodi riescano comunque a coordinarsi automaticamente.
Sicuramente il protocollo è pero semplice (requisito 4).

\end{document}
